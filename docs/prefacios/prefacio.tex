% !TeX root = ../proyecto.tex

\chapter*{}
%\thispagestyle{empty}
%\cleardoublepage

%\thispagestyle{empty}


\cleardoublepage
\thispagestyle{empty}

\begin{center}
       {\large\bfseries Algoritmos meméticos para reducir datos de entrenamiento en modelos de aprendizaje profundo
              convolucionales}\\
\end{center}
\begin{center}
       José Ruiz López (alumno)\\
\end{center}

%\vspace{0.7cm}
\noindent{\textbf{Palabras clave}: Algoritmos meméticos, Imágenes, Modelos de Aprendizaje profundo convolucionales}\\

\vspace{0.7cm}
\noindent{\textbf{Resumen}}\\

Los \textbf{modelos de Aprendizaje Profundo} (Deep Learning) han supuesto un verdadero hito en la
\textbf{Inteligencia Artificial}, ya que son capaces de procesar grandes volúmenes de datos, además de reconocer
patrones sumamente complejos.
Dentro de estos, los \textbf{modelos convolucionales} se han destacado como particularmente efectivos a la hora de
identificar objetos y características en imágenes —una capacidad esencial para muchas aplicaciones modernas—.
Sin embargo, a diferencia de los seres humanos, estos modelos requieren una gran cantidad de datos de
entrenamiento para cada categoría que deben aprender.
Esto implica un proceso de entrenamiento más largo y, muchas veces, la recolección de los datos necesarios puede ser
problemática, según el tipo de información que se necesite.

Además de la dificultad en la obtención de datos,  las nuevas normativas europeas en torno a la inteligencia artificial
establecen la necesidad de auditar no solo los modelos, sino también los datos
utilizados para entrenarlos, especialmente cuando se trata de aplicaciones de IA que manejan datos sensibles.
Estas auditorías, por su propia naturaleza, se volverán más complejas conforme aumente el tamaño del conjunto de
entrenamiento.
Por lo tanto, se vuelve completamente necesario desarrollar estrategias que permitan
\textbf{reducir el tamaño de los conjuntos de datos de entrenamiento} intentado comprometer la calidad del modelo lo mínimo posible.

En este trabajo, proponemos el uso de \textbf{algoritmos meméticos} para establecer un proceso de reducción del conjunto de
\textbf{entrenamiento}, lo que se conoce como \textbf{selección de instancias}.
La idea es seleccionar un conjunto reducido de imágenes representativas que, junto con las técnicas de aumento de
datos, sean suficientes para entrenar modelos convolucionales de manera óptima.
De este modo, se podría reducir significativamente el tamaño del conjunto de entrenamiento, manteniendo la calidad
del aprendizaje y, a su vez, facilitando tanto el proceso de auditoría como la eficiencia computacional del sistema.


\cleardoublepage


\thispagestyle{empty}


\begin{center}
       {\large\bfseries Memetic Algorithms for Reducing Training Data in Convolutional Deep Learning Models}\\
\end{center}
\begin{center}
       José, Ruiz López (student)\\
\end{center}

%\vspace{0.7cm}
\noindent{\textbf{Keywords}: Memetic Algorithms, Images, Convolutional Deep Learning Models}\\

\vspace{0.7cm}
\noindent{\textbf{Abstract}}\\

\textbf{Deep Learning models} have marked a true milestone in the field of \textbf{Artificial Intelligence}, as they are capable of 
processing large volumes of data and recognizing highly complex patterns.
Among these, \textbf{convolutional models} have stood out as particularly effective in identifying objects and
features in images—an essential capability for many modern applications.
However, unlike humans, these models require a large amount of training data for each category they need to learn.
This implies a longer training process, and in many cases, collecting the necessary data can be problematic depending
on the type of information required.

In addition to the difficulty of obtaining data, new European regulations on artificial intelligence establish the need to 
audit not only the models but also the data used to train them, especially in AI applications that handle sensitive data.
These audits, by their very nature, will become increasingly complex as the size of the training set grows.
Therefore, it becomes essential to develop strategies that allow for \textbf{reducing the size of training datasets}, 
while minimizing any compromise in model quality.

In this work, we propose the use of \textbf{memetic algorithms} to implement a \textbf{training data reduction process}, 
also known as \textbf{instance selection}.
The idea is to select a small set of representative images that, along with data augmentation techniques, 
are sufficient to optimally train convolutional models.
In this way, it would be possible to significantly reduce the size of the training set while maintaining learning quality and, 
at the same time, facilitating both the auditing process and the system's computational efficiency.

\chapter*{}
\thispagestyle{empty}

\noindent\rule[-1ex]{\textwidth}{2pt}\\[4.5ex]

Yo, \textbf{José Ruiz López}, alumno de la titulación INGENIERÍA INFORMÁTICA de la \textbf{Escuela Técnica Superior
       de Ingenierías Informática y de Telecomunicación de la Universidad de Granada}, con DNI \textbf{77964364E}, autorizo la
ubicación de la siguiente copia de mi Trabajo Fin de Grado en la biblioteca del centro para que pueda ser
consultada por las personas que lo deseen.

\vspace{6cm}

\noindent Fdo: José Ruiz López

\vspace{2cm}

\begin{flushright}
       Granada a \today.
\end{flushright}


\chapter*{}
\thispagestyle{empty}

\noindent\rule[-1ex]{\textwidth}{2pt}\\[4.5ex]

D. \textbf{Daniel Molina Cabrera (tutor}, Profesor del Departamento Ciencias de la Computación e Inteligencia
Artificial de la Universidad de Granada.

\vspace{0.5cm}

\textbf{Informan:}

\vspace{0.5cm}

Que el presente trabajo, titulado \textit{\textbf{Algoritmos meméticos para reducir datos de entrenamiento en modelos de aprendizaje profundo convolucionales}},
ha sido realizado bajo su supervisión por \textbf{José Ruiz López (alumno)}, y autorizamos la defensa de dicho trabajo  ante el tribunal que corresponda.

\vspace{0.5cm}

Y para que conste, expiden y firman el presente informe en Granada a \today.

\vspace{1cm}

\textbf{Los directores:}

\vspace{5cm}

\noindent \textbf{Daniel Molina Cabrera (tutor)}

\chapter*{Agradecimientos}
\thispagestyle{empty}

\vspace{1cm}


Quiero expresar mi más sincero agradecimiento a todas las personas que han hecho posible la realización de este Trabajo de Fin de Grado.

En primer lugar, me gustaría agradecer al profesor Daniel Molina Cabrera, mi tutor, por su valiosa guía, por su apoyo continuo durante 
todo el proceso y por brindarme acceso a los recursos necesarios para llevar a cabo esta investigación.
Su experiencia y disponibilidad han sido fundamentales para que este proyecto pudiera desarrollarse de forma rigurosa y enriquecedora.

También quiero destacar lo desafiante que ha sido compaginar este trabajo académico con mis responsabilidades laborales.
Ha requerido un esfuerzo constante y una gran capacidad de organización, pero también me ha permitido valorar aún más el proceso y el aprendizaje adquirido.

Agradezco de corazón a mis padres y amigos por su comprensión, paciencia y apoyo emocional durante los momentos más exigentes del camino.
Y, sobre todo, a mi pareja, sin ella no habría sido posible superar el reto que supone realizar un trabajo como este. 
Su apoyo incondicional, su confianza en mí y la calma que me ha transmitido en los momentos más difíciles han sido clave para llegar hasta el final.

Asimismo, extiendo mi gratitud a todas las personas y compañeros que, directa o indirectamente, han contribuido con sus comentarios, sugerencias y tiempo.
Gracias a ellos, este trabajo ha podido alcanzar una mayor profundidad y claridad.

Por último, quiero agradecer a todas aquellas comunidades de código abierto y herramientas libres que han permitido que este proyecto se 
desarrollara sin barreras tecnológicas, facilitando el aprendizaje y la innovación.
