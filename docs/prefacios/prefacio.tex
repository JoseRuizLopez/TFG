\chapter*{}
%\thispagestyle{empty}
%\cleardoublepage

%\thispagestyle{empty}


\cleardoublepage
\thispagestyle{empty}

\begin{center}
{\large\bfseries Algoritmos meméticos para reducir datos de entrenamiento en modelos de aprendizaje profundo
convolucionales}\\
\end{center}
\begin{center}
José Ruiz López (alumno)\\
\end{center}

%\vspace{0.7cm}
\noindent{\textbf{Palabras clave}: Algoritmos meméticos, Imágenes, Modelos de Aprendizaje profundo convolucionales}\\

\vspace{0.7cm}
\noindent{\textbf{Resumen}}\\

Los modelos de \textbf{Aprendizaje Profundo} (Deep Learning) han supuesto un hito en la Inteligencia Artificial al ser
capaz de procesar y ser capaces de reconocer patrones complejos.
Dentro de estos, los modelos convolucionales se han mostrado muy capaces de identificar todo tipo de objetos/
características en imágenes.\\[6pt]

Sin embargo, a diferencia de las personas, requieren un número muy alto de datos de entrenamiento para cada categoría
que debe aprender.
Eso implica, además de entrenamiento más largo, una recogida de datos de entrenamiento que, según lo que se desea que
aprenda, puede ser problemático de obtener.
Además de la obtención de los datos, la nueva ley europea sobre IA, \textbf{IA Act} requerirá sobre aplicaciones de IA
con datos sensibles, una auditoría no solo del propio modelo, sino también de los datos utilizados para entrenarla.
Auditoría que crecerá en complejidad confirme aumente en número el conjunto de entrenamiento.
Por tanto, se hace conveniente poder reducir el conjunto de entrenamiento.\\[6pt]

Ya se ha confirmado que incrementar el número de imágenes de entrenamiento puede mejorar el proceso o no, según si las
imágenes realmente contribuyan al proceso de entrenamiento.
Es más, gracias a las técnicas de aumento de datos (\textbf{Data Augmentation}) la posible necesidad de imágenes muy
similares entre sí se reduce al ser capaz de construirse de forma automática más imágenes de entrenamiento (imágenes
que no suponen un problema de cara a una autoría).\\[6pt]

En este trabajo planteamos el uso de estrategias avanzadas, como \textbf{algoritmos metaheurísticos}, y métricas de
similitud entre imágenes, para establecer un proceso de reducción de imágenes de entrenamiento (selección de
instancias) para poder reducir el conjunto de entrenamiento.
De esta manera, se seleccionarían solo un conjunto reducido de imágenes representativas que, gracias a las
\textbf{técnicas de aumento de datos}, puedan entrenar modelos con una calidad suficiente.
Por lo tanto, se podría reducir muy significativamente el conjunto de entrenamiento.\\[6pt]
\cleardoublepage


\thispagestyle{empty}


\begin{center}
{\large\bfseries Memetic Algorithms for Reducing Training Data in Convolutional Deep Learning Models}\\
\end{center}
\begin{center}
José, Ruiz López (student)\\
\end{center}

%\vspace{0.7cm}
\noindent{\textbf{Keywords}: Memetic Algorithms, Images, Convolutional Deep Learning Models}\\

\vspace{0.7cm}
\noindent{\textbf{Abstract}}\\

\textbf{Deep Learning} models have marked a milestone in Artificial Intelligence by being able to process and recognize
complex patterns.
Among these, convolutional models have proven very capable of identifying all kinds of objects/features in images.\\
[6pt]

However, unlike humans, they require a very high number of training data for each category they need to learn.
This implies not only longer training times but also a data collection process that can be problematic to obtain,
depending on what is desired for the model to learn.
In addition to data acquisition, the new European law on AI, the \textbf{AI Act}, will require an audit not only of the
model itself but also of the data used to train it, especially when dealing with sensitive data.
The complexity of this audit will grow as the size of the training dataset increases.
Therefore, it becomes necessary to reduce the training dataset.\\[6pt]

It has been confirmed that increasing the number of training images may or may not improve the process, depending on
whether the images truly contribute to the training process.
Moreover, thanks to \textbf{data augmentation techniques}, the potential need for very similar images is reduced, as it
is possible to automatically generate more training images that do not pose authorship issues.\\[6pt]

In this work, we propose the use of advanced strategies, such as \textbf{metaheuristic algorithms} and image similarity
metrics, to establish a process for reducing training images (instance selection) in order to minimize the training
dataset.
This way, only a reduced set of representative images would be selected, which, thanks to \textbf{data augmentation
techniques}, could sufficiently train models.
Thus, the training dataset could be significantly reduced.\\[6pt]

\chapter*{}
\thispagestyle{empty}

\noindent\rule[-1ex]{\textwidth}{2pt}\\[4.5ex]

Yo, \textbf{José Ruiz López}, alumno de la titulación TITULACIÓN de la \textbf{Escuela Técnica Superior
de Ingenierías Informática y de Telecomunicación de la Universidad de Granada}, con DNI 77964364E, autorizo la
ubicación de la siguiente copia de mi Trabajo Fin de Grado en la biblioteca del centro para que pueda ser
consultada por las personas que lo deseen.

\vspace{6cm}

\noindent Fdo: José Ruiz López

\vspace{2cm}

\begin{flushright}
Granada a X de mes de 201 .
\end{flushright}


\chapter*{}
\thispagestyle{empty}

\noindent\rule[-1ex]{\textwidth}{2pt}\\[4.5ex]

D. \textbf{Daniel Molina Cabrera (tutor}, Profesor del Departamento Ciencias de la Computación e Inteligencia
Artificial de la Universidad de Granada.

\vspace{0.5cm}

\textbf{Informan:}

\vspace{0.5cm}

Que el presente trabajo, titulado \textit{\textbf{Algoritmos meméticos para reducir datos de entrenamiento en modelos
de aprendizaje profundo convolucionales}},
ha sido realizado bajo su supervisión por \textbf{José Ruiz López (alumno)}, y autorizamos la defensa de dicho trabajo
ante el tribunal que corresponda.

\vspace{0.5cm}

Y para que conste, expiden y firman el presente informe en Granada a X de mes de 201 .

\vspace{1cm}

\textbf{Los directores:}

\vspace{5cm}

\noindent \textbf{Daniel Molina Cabrera (tutor)}

\chapter*{Agradecimientos}
\thispagestyle{empty}

       \vspace{1cm}


Poner aquí agradecimientos...

