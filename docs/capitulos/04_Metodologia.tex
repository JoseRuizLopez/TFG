\chapter{Metodología}\label{ch:Metodologia}
En este capítulo se detallan la arquitectura del sistema que se ha implementado, como la tecnología usada, diseños, etc.
\\[6pt]

El repositorio ha sido almacenado en la plataforma \textbf{Github} desde el comienzo del proyecto.
De forma obvia, la herramienta usada para el control de las versiones es \textbf{Git}~\cite{chacon2014pro}. \\[6pt]

\section{Descripción del Sistema}\label{sec:descripcion_del_sistema}
%Descripción del Sistema: Detalla la arquitectura del sistema que estás implementando.
La estructura del proyecto es la siguiente:
\begin{itemize}
    \item \texttt{data} -- Almacena los dataset utilizados por el proyecto.
    \item \texttt{docs} -- Documentación del proyecto en latex.
    \item \texttt{img} -- Imágenes generadas en el proyecto.
    \item \texttt{LICENSE} -- Términos de distribución del proyecto.
    \item \texttt{README.md} -- Descripción general.
    \item \texttt{requirements.txt} -- Dependencias del proyecto.
    \item \texttt{results} -- Resultados generados por el proyecto (fitness, tiempos, etc.).
    \item \texttt{scripts} -- Scripts y programas secundarios para ejecutarse en el servidor GPU\@.
    \item \texttt{src} -- Código fuente del proyecto.
    \item \texttt{utils} -- Módulos y scripts de utilidad.
%    \item \texttt{venv} -- Entorno virtual de Python.
\end{itemize}

\section{Herramientas y Lenguajes de Programación}\label{sec:herramientas_y_lenguajes_de_programacion}
%Herramientas y Lenguajes de Programación: Lista las herramientas y tecnologías que usarás.
El proyecto ha sido programado usado el lenguaje de programación  \textbf{Python 3.10}.
Junto a las librerias de \textbf{PyTorch 2.3.1}, \textbf{scikit-learn1.5.2} y \textbf{numpy 2.0.0}. \\[6pt]

\section{Proceso de Desarrollo}\label{sec:proceso_de_desarrollo}
%Proceso de Desarrollo: Explica cómo planeas programar y realizar pruebas.
Al plantear el TFG, se pensó en como desarrollarlo. \\[6pt]

Para ello se definieron unos objetivos inciales y un plan temporal de cuyo desarrollo.
Tras el planteamiento inicial, se realizó una investigación preliminar para tener contexto del Estado del arte y de
que herramientes y tecnologias usar. \\[6pt]

Una vez planeado todo, se empezó a desarrollar código, junto a pruebas iniciales.
Despúes de una implementación inicial, se desarrollaría una experimentación iterativa para añadir nuevos experimentos y
realizar ajustes continuos.
Y finalmente realizar la documentación del TFG, junto a la presentación y defensa del proyecto final.




%EXTRA
%Nuevas Pruebas/Experimentos: Documenta las pruebas adicionales que decidiste realizar, incluyendo la justificación.
%Resultados y Ajustes: Incluye los resultados de estas nuevas pruebas y los ajustes realizados en consecuencia.