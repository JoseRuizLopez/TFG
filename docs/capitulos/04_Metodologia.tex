\chapter{Metodología}\label{ch:Metodologia}
En este capítulo se detallan la arquitectura del sistema que se ha implementado, como la tecnología usada, diseños, etc.
\\[6pt]

El repositorio ha sido almacenado en la plataforma \textbf{Github} desde el comienzo del proyecto.
De forma obvia, la herramienta usada para el control de las versiones es \textbf{Git}~\cite{}. \\[6pt]

\section{Descripción del Sistema}\label{sec:descripcion_del_sistema}
%Descripción del Sistema: Detalla la arquitectura del sistema que estás implementando.
La estructura del proyecto es la siguiente:
\begin{itemize}
    \item \texttt{data} -- Almacena los dataset utilizados por el proyecto.
    \item \texttt{docs} -- Documentación del proyecto en latex.
    \item \texttt{img} -- Imágenes generadas en el proyecto.
    \item \texttt{LICENSE} -- Términos de distribución del proyecto.
    \item \texttt{README.md} -- Descripción general.
    \item \texttt{requirements.txt} -- Dependencias del proyecto.
    \item \texttt{results} -- Resultados generados por el proyecto (fitness, tiempos, etc.).
    \item \texttt{scripts} -- Scripts y programas secundarios para ejecutarse en el servidor GPU\@.
    \item \texttt{src} -- Código fuente del proyecto.
    \item \texttt{utils} -- Módulos y scripts de utilidad.
\end{itemize}

\section{Herramientas y Lenguajes de Programación}\label{sec:herramientas_y_lenguajes_de_programacion}
%Herramientas y Lenguajes de Programación: Lista las herramientas y tecnologías que usarás.
El desarrollo del proyecto se ha llevado a cabo utilizando \textbf{Python 3.10}~\cite{} como lenguaje principal,
debido a su versatilidad y amplia adopción en el campo del \textbf{aprendizaje profundo} y la
\textbf{manipulación de datos}.
Python es conocido por su facilidad de uso, extensibilidad y la gran cantidad de bibliotecas disponibles para el
procesamiento de datos y la implementación de modelos de \textbf{machine learning}. \\[6pt]

Las principales bibliotecas empleadas durante el desarrollo son las siguientes:
\begin{itemize}
    \item \textbf{PyTorch 2.3.1}~\cite{}: Biblioteca usada para implementar redes neuronales y modelos de aprendizaje
profundo, con flexibilidad para experimentar y aprovechar GPU\@.
    \item \textbf{Scikit-learn 1.5.2}~\cite{}: Librería para el procesamiento de datos y modelos de machine learning,
que incluye herramientas para selección de características y validación cruzada.
    \item \textbf{Numpy 2.0.0}~\cite{}: Herramienta para realizar operaciones matemáticas y de álgebra lineal, esencial
para manejar grandes conjuntos de datos.
    \item \textbf{Polars 1.9.0}~\cite{}: Alternativa más eficiente a Pandas, utilizada para gestionar grandes volúmenes
de datos y mejorar el procesamiento de DataFrames.
\end{itemize}

Cada una de estas herramientas fue seleccionada por su robustez y su idoneidad para cumplir con los requisitos
específicos del proyecto, facilitando tanto la implementación de los algoritmos meméticos como la reducción y el
análisis de los datos utilizados en los modelos de aprendizaje profundo. \\[6pt]

\section{Proceso de Desarrollo}\label{sec:proceso_de_desarrollo}
%Proceso de Desarrollo: Explica cómo planeas programar y realizar pruebas.
El desarrollo de este Trabajo de Fin de Grado (TFG) se inició con una cuidadosa \textbf{planificación inicial}.
Se definieron \textbf{objetivos claros} y un \textbf{plan temporal} que guiaría cada una de las fases del proyecto.
El primer paso consistió en llevar a cabo una \textbf{investigación preliminar} para comprender el
\textbf{estado del arte} en cuanto al uso de algoritmos meméticos en el ámbito del \textbf{aprendizaje profundo}.
Esta investigación incluyó la revisión de trabajos previos relevantes, así como la identificación de las
\textbf{herramientas y tecnologías} más adecuadas para el desarrollo del proyecto. \\[6pt]

Con una base teórica sólida y una planificación definida, se procedió a la \textbf{fase de desarrollo}.
Esta etapa comenzó con la implementación de un \textbf{prototipo inicial}, que permitió realizar pruebas preliminares y
sentar las bases del código principal.
A medida que se implementaba esta primera versión, se llevó a cabo una \textbf{experimentación iterativa}, un proceso
en el que se fueron incorporando \textbf{nuevos experimentos} y realizando \textbf{ajustes continuos} a partir de los
resultados obtenidos.
Este enfoque de \textbf{desarrollo incremental} permitió refinar el rendimiento de los algoritmos meméticos y mejorar la
eficiencia de los modelos. \\[6pt]

Finalmente, el proceso culminó con la \textbf{documentación exhaustiva} de todos los pasos realizados, así como la
\textbf{preparación de la presentación} y la \textbf{defensa} del proyecto final.
La documentación incluyó tanto el \textbf{análisis de los resultados experimentales} como una reflexión crítica sobre
los logros y las limitaciones del TFG, asegurando que todo el proceso fuera presentado de manera clara y coherente.
\\[6pt]





%EXTRA
%Nuevas Pruebas/Experimentos: Documenta las pruebas adicionales que decidiste realizar, incluyendo la justificación.
%Resultados y Ajustes: Incluye los resultados de estas nuevas pruebas y los ajustes realizados en consecuencia.