% !TeX root = ../proyecto.tex

\chapter{Introducción}\label{ch:introduccion}

\section{Contexto}\label{sec:contexto}
%Contexto: Breve introducción al tema y su relevancia.
En la actualidad, vivimos en una era marcada por una constante y acelerada generación de datos.
Este fenómeno ha incrementado la necesidad de desarrollar métodos eficaces para el procesamiento y análisis de grandes
volúmenes de información.
En este contexto, los \textbf{modelos de aprendizaje profundo}, y en particular las
\textbf{redes neuronales convolucionales} (CNN), han demostrado un notable rendimiento en tareas como la
\textbf{clasificación de imágenes}, el \textbf{reconocimiento de patrones} y diversas aplicaciones de alta complejidad.
No obstante, el entrenamiento de estos modelos suele requerir grandes cantidades de datos, lo que plantea desafíos
significativos tanto en términos de \textbf{tiempo} como de \textbf{costes} asociados a su obtención.



Conforme los sistemas de inteligencia artificial evolucionan hacia estructuras más sofisticadas y precisas, la
disponibilidad de conjuntos de datos amplios y adecuados se vuelve un requisito cada vez más crucial.
Sin embargo, la recopilación, almacenamiento y tratamiento de estos datos suponen obstáculos importantes, especialmente
para aquellas instituciones u organizaciones que cuentan con recursos limitados.
Esta situación pone de relieve la necesidad de investigar estrategias innovadoras que permitan
\textbf{reducir y optimizar los conjuntos de datos} sin comprometer el rendimiento de los modelos entrenados.


\section{Motivación}\label{sec:motivacion}
La necesidad de \textbf{reducir los conjuntos de datos de entrenamiento} responde al objetivo de mejorar la
\textbf{eficiencia} en el desarrollo de modelos de aprendizaje profundo.
Aunque las redes neuronales convolucionales han demostrado un rendimiento notable en diversas tareas, su implementación
conlleva \textbf{altos costes computacionales} y una gran demanda de datos, lo que representa una barrera considerable
para muchos entornos, especialmente aquellos con recursos limitados.
Una estrategia prometedora consiste en entrenar estos modelos utilizando únicamente una fracción de los datos
disponibles, seleccionados de manera óptima.
Esta aproximación permitiría disminuir significativamente el consumo de recursos sin comprometer la precisión del
modelo, lo que supondría un avance importante para la \textbf{inteligencia artificial}, en particular en aplicaciones
donde existen \textbf{restricciones de recursos}.


En este contexto, la selección de subconjuntos representativos se presenta como una solución eficaz para reducir tanto
los tiempos de entrenamiento como el uso de recursos, sin afectar negativamente el rendimiento.
Es precisamente en este punto donde las \textbf{metaheurísticas} adquieren un papel relevante.
Estas técnicas de optimización están diseñadas para abordar problemas complejos en los que los métodos tradicionales
resultan ineficaces, gracias a sus \textbf{estrategias de búsqueda} y \textbf{exploración del espacio de soluciones}.
Al combinar diversas heurísticas, permiten encontrar soluciones aproximadas en tiempos razonables, lo que las convierte
en una herramienta especialmente útil cuando la obtención de una solución exacta resulta inviable desde el punto de
vista computacional.


Por tanto, las metaheurísticas se posicionan como una alternativa sólida para mejorar la eficiencia y accesibilidad del
aprendizaje profundo, incluso en contextos con recursos limitados.
Esto resulta fundamental para avanzar hacia una inteligencia artificial más abierta, \textbf{democrática} y aplicable
en una mayor variedad de escenarios.


Este Trabajo de Fin de Grado tiene como objetivo aplicar técnicas metaheurísticas para realizar una selección
inteligente de ejemplos, con el fin de reducir el tamaño de los conjuntos de entrenamiento sin que ello afecte
significativamente a los resultados obtenidos.
La investigación aspira a contribuir al desarrollo de modelos más \textbf{eficientes}, accesibles y económicamente
sostenibles, fomentando así un futuro en el que la inteligencia artificial sea más \textbf{inclusiva} y
\textbf{sostenible}.


\section{Objetivos}\label{sec:objetivos}
%Objetivos: Especifica qué pretendes lograr con tu TFG.
El objetivo principal de este TFG es investigar la aplicación de \textbf{metaheurísticas} para la
\textbf{reducción de conjuntos de datos de entrenamiento} en modelos de \textbf{aprendizaje profundo convolucionales}.
Este estudio permitirá evaluar el impacto de dichos algoritmos en la \textbf{eficiencia computacional} y en el
\textbf{rendimiento de los modelos}.


Para cumplir con este objetivo general, se plantean los siguientes \textbf{objetivos específicos}:

\begin{itemize}
      \item \textbf{Desarrollar} e implementar metaheurísticas que busquen conjuntos reducidos de datos de
            entrenamiento, con el fin de reducir el volumen de datos requerido para entrenar modelos convolucionales —sin
            comprometer la precisión de los resultados—.
      \item \textbf{Evaluar} el impacto de la reducción de datos en el rendimiento de los modelos, comparando aspectos
            clave como la precisión, eficacia y el tiempo de entrenamiento —en modelos entrenados con conjuntos de datos completos
            frente a conjuntos reducidos—.
      \item \textbf{Mejorar la eficiencia} del entrenamiento de redes neuronales convolucionales mediante el uso de
            metaheurísticas, analizando los beneficios en términos de reducción de tiempo y costo computacional.
      \item \textbf{Contribuir al avance} de soluciones innovadoras en el campo del aprendizaje profundo, especialmente
            en escenarios con limitaciones de datos; facilitando así el acceso a esta tecnología a sectores que, de otro modo,
            tendrían dificultades para implementarla de manera efectiva.
\end{itemize}


A través de este estudio, se busca no solo mejorar el rendimiento y la eficiencia de los modelos convolucionales, sino
también fomentar el desarrollo de soluciones más \textbf{sostenibles y accesibles} en el ámbito de la inteligencia
artificial.
