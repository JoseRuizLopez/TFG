\chapter{Introducción}\label{ch:introduccion}

\section{Contexto}\label{sec:contexto}
%Contexto: Breve introducción al tema y su relevancia.
En la era actual, donde los datos se generan a un ritmo vertiginoso, la necesidad de procesar y analizar grandes
volúmenes de información se ha vuelto crucial.
Los \textbf{modelos de aprendizaje profundo}, y en particular las \textbf{redes neuronales convolucionales}, han
demostrado su capacidad para alcanzar niveles sin precedentes de precisión en tareas como la clasificación de imágenes
y el reconocimiento de patrones.
Sin embargo, estos modelos a menudo requieren enormes cantidades de datos de entrenamiento, lo que plantea desafíos
significativos en términos de tiempo, costo y recursos computacionales. \\[6pt]

Es aquí donde entra en juego la importancia de los \textbf{algoritmos meméticos}.
Combinando las fortalezas de las técnicas evolutivas con \textbf{métodos de búsqueda local}, estos algoritmos ofrecen
un enfoque innovador y eficiente para la reducción de datos.
La optimización de conjuntos de datos no solo puede mejorar el rendimiento de los modelos, sino que también permite una
capacitación más rápida y menos costosa, facilitando así la investigación y el desarrollo en diversas aplicaciones.
\\[6pt]

Realizar un TFG sobre este tema no solo representa una oportunidad para explorar una frontera apasionante de la
inteligencia artificial.
La investigación en algoritmos meméticos para la reducción de datos puede ser clave para hacer más accesible el
\textbf{aprendizaje profundo} a aquellos que enfrentan limitaciones de datos, permitiendo asi un futuro donde la
inteligencia artificial sea más inclusiva y eficiente. \\[6pt]

\section{Objetivos}\label{sec:objetivos}
%Objetivos: Especifica qué pretendes lograr con tu TFG.
\section{Metodología}\label{sec:metodologia}
%Metodología: Describe brevemente el enfoque que seguirás.
