\chapter{Metodología y Presupuesto}\label{ch:metodologia-y-presupuesto}
\section{Metodología}\label{sec:metodologia}
La metodología adoptada para el desarrollo de este proyecto ha estado inspirada en el marco ágil
\textbf{Scrum}~\cite{}, una estrategia ampliamente utilizada en proyectos que requieren flexibilidad,
adaptabilidad y entregas incrementales.
Dado que un trabajo de investigación como este demanda constantes ajustes en función de los hallazgos y resultados
obtenidos, Scrum ha sido el marco adecuado para organizar el desarrollo de manera iterativa y colaborativa, facilitando
la mejora continua durante el proceso.

\subsection{Metodología Ágil Inspirada en Scrum}\label{subsec:metodologia-agil-inspirada-en-scrum}
El desarrollo del proyecto se organizó en \textbf{sprints}, que consistían en ciclos cortos de
trabajo (de dos semanas, generalmente), donde se definían objetivos concretos al inicio y se revisaban los avances al
final de cada sprint.
Esto permitió adaptar el trabajo de acuerdo con los descubrimientos y problemas emergentes a lo largo del proyecto.
\\[6pt]

Además, el uso de \textbf{Git}~\cite{} como herramienta para el control de versiones fue clave en la gestión
de los avances de desarrollo.
Git permitió mantener un registro preciso de cada cambio en el código y los algoritmos, facilitando la colaboración con
el tutor y garantizando una trazabilidad eficiente de los ajustes realizados durante cada iteración.
Gracias a Git, se pudo revisar, deshacer y modificar el código de forma ágil, integrando nuevos enfoques o mejoras de
manera controlada tras cada sprint.

\subsection{Ciclos de Trabajo}\label{subsec:ciclos-de-trabajo}
En consonancia con Scrum, cada ciclo de trabajo (o sprint) incluyó una planificación detallada, desarrollo, pruebas y
análisis de los resultados obtenidos.
Los sprints se estructuraron en torno a objetivos específicos, como la implementación de un algoritmo o el ajuste de
parámetros, y estuvieron formados por las siguientes fases:

\begin{itemize}
    \item \textbf{Planificación del sprint}: Al inicio de cada sprint, se realizaba una sesión de planificación donde
    se establecían los objetivos específicos del ciclo.
    Estos objetivos podían incluir tareas como la implementación de un nuevo algoritmo o la optimización de uno
    existente.
    La planificación se ajustaba constantemente en función del progreso logrado en sprints anteriores y de los nuevos
    descubrimientos.
    Aquí también se establecían las prioridades para asegurar que los esfuerzos se enfocaran en los puntos críticos del
    proyecto.
    \item \textbf{Desarrollo e implementación de algoritmos}: Durante la fase de desarrollo, se implementaban los
    algoritmos planificados, como los algoritmos aleatorios, de búsqueda local y genéticos.
    En lugar de completar el desarrollo de un algoritmo en su totalidad antes de pasar al siguiente, se seguía un
    enfoque iterativo.
    Esto permitía construir versiones básicas de los algoritmos que luego se refinaban en iteraciones sucesivas.


    Git fue fundamental en esta etapa, ya que permitía gestionar los cambios en el código de manera efectiva.
    A medida que se introducían mejoras o correcciones, Git ayudaba a mantener el historial de versiones, facilitando
    la identificación de regresiones o la comparación entre diferentes enfoques.
    \item \textbf{Pruebas y ajustes iterativos}: Una vez implementados los algoritmos, se realizaban pruebas continuas
    para evaluar su rendimiento en función de los resultados obtenidos.
    Esta fase era crucial para identificar posibles áreas de mejora en los algoritmos o en los parámetros utilizados.
    Los resultados de las pruebas proporcionaban retroalimentación inmediata, lo que permitía realizar ajustes en
    tiempo real dentro del mismo sprint.
    Durante este ciclo, los algoritmos se ajustaban y optimizaban iterativamente, lo que permitía identificar errores o
    ineficiencias de manera temprana.
    \item \textbf{Revisión y análisis de resultados}: Al final de cada sprint, se realizaba una revisión detallada de
    los resultados obtenidos durante las pruebas.
    Este análisis incluía no solo la evaluación del rendimiento de los algoritmos, sino también la reflexión sobre los
    avances logrados respecto a los objetivos iniciales del sprint.


    Durante las reuniones con el tutor, se discutían los resultados y se revisaban posibles cambios en la planificación
    de los próximos sprints, ajustando las prioridades y estableciendo nuevos objetivos en función de los hallazgos.
    Esta fase permitía un enfoque incremental, donde las mejoras y ajustes se incorporaban de manera continua, lo que
    aseguraba que el proyecto avanzara de forma óptima y se pudieran hacer correcciones oportunas antes de continuar
    con el siguiente sprint.

    \item \textbf{Documentación de los algoritmos y resultados}: El último paso en cada sprint consistía en documentar
    los avances alcanzados, tanto en la implementación de los algoritmos como en los resultados obtenidos.
    Esta documentación no solo reflejaba el estado del proyecto al final de cada sprint, sino que también era esencial
    para realizar un seguimiento exhaustivo de las mejoras realizadas en cada iteración.
\end{itemize}

\subsection{Git como Herramienta de Control de Versiones}\label{subsec:git-como-herramienta-de-control-de-versiones}
\textbf{Git} se utilizó no solo como repositorio de control de versiones, sino también para gestionar de manera
eficiente el flujo de trabajo entre el servidor y el entorno local. \\[6pt]

Esto permitió:
\begin{itemize}
    \item \textbf{Sincronizar} el código y los datos entre el alumno y el servidor, garantizando que siempre se
    trabajara con la última versión del proyecto.
    \item Mantener un \textbf{historial de cambios}, facilitando la posibilidad de revertir versiones si era necesario
    y permitiendo comparaciones entre diferentes iteraciones.
    \item \textbf{Colaborar} de manera eficiente con el tutor, quien podía revisar el código actualizado al final de
    cada sprint, realizar comentarios y sugerir cambios en cada versión.
\end{itemize}

\subsection{Adaptación del Tiempo de los Sprints}\label{subsec:adaptacion-del-tiempo-de-los-sprints}
Debido a ciertos retrasos en el desarrollo del TFG, fue necesario ajustar la duración de los sprints para asegurarse de
que todas las fases y pruebas del proyecto se completaran a tiempo.
La flexibilidad de Scrum y el uso de Git para coordinar versiones y tareas pendientes permitieron reorganizar las
prioridades y garantizar que los objetivos del proyecto fueran alcanzados sin comprometer la calidad de los resultados.

\section{Presupuesto}\label{sec:presupuesto}
El presupuesto estimado incluye tanto el tiempo dedicado al proyecto (mano de obra) como los recursos computacionales y
otros gastos relacionados.
Aunque muchos recursos utilizados son gratuitos o de bajo coste, se ha realizado una estimación considerando el valor
del tiempo invertido y el uso de recursos computacionales.

\subsection{Mano de obra}\label{subsec:mano-de-obra}
El trabajo total estimado es de \textbf{400 horas}, repartidas de manera flexible a lo largo de los ciclos.
El coste por hora se ha estimado en \textbf{20 euros}~\cite{salario-medio}, lo cual incluye las tareas de investigación, desarrollo de
algoritmos, análisis de resultados y documentación.

\begin{table}[htp]\label{tab:mano-de-obra}
    \centering
    \begin{adjustbox}{width=\linewidth}
        \begin{tabular}{|l|c|c|c|}
            \hline
            \textbf{Ciclo de trabajo} & \textbf{Horas dedicadas} & \textbf{Coste por hora (€)} & \textbf{Coste total (€)} \\ \hline
            Planificación del sprint & 30 & 20 & 600 \\
            Desarrollo e implementación de algoritmos & 80 & 20 & 1.600 \\
            Pruebas y ajustes iterativos & 180 & 20 & 3.600 \\
            Revisión y análisis de resultados & 55 & 20 & 1.100 \\
            Documentación de algoritmos y resultados & 55 & 20 & 1.100 \\ \hline
            \textbf{Total} & \textbf{400 horas} & & \textbf{8.000 €} \\ \hline
        \end{tabular}
    \end{adjustbox}
    \caption{Coste estimado de la mano de obra basado en los ciclos de trabajo.}
\end{table}



\subsection{Recursos computacionales}\label{subsec:recursos-computacionales}
El proyecto ha requerido el uso de recursos computacionales para entrenar los modelos de aprendizaje profundo,
especialmente para evaluar la eficacia de los algoritmos en la reducción de datos. \\[6pt]

El tutor \textbf{Daniel Molina Cabrera} me puso en disposición el acceso a un servidor de investigación con
disponibilidad de GPU, de manera que no ha supuesto ningún coste.
Por ello, para suponer un coste estimado, se ha supuesto lo que nos costaria un servidor de
\textbf{Google Cloud}~\cite{} de \textbf{Compute Engine}~\cite{} en Bélgica. \\[6pt]

Los componentes equivalentes del servidor en Compute Engine serían:
\begin{itemize}
    \item Un \textbf{Intel Xeon E-2226G} equivaldría a un \textbf{c2-standard-8}, cuyo coste es de 0.43 EUR/h.
    \item Un \textbf{NVIDIA TITAN Xp} equivaldría a una \textbf{NVIDIA K80 1 GPU}, GDDR5 de 12 GB cuyo coste es de 0.42
EUR/h.
\end{itemize}

De manerá que los gastos estimados serían:
\begin{table}[htp]\label{tab:recursos-computacionales}
    \centering
    \begin{adjustbox}{width=\linewidth}
        \begin{tabular}{|l|c|c|c|}
            \hline
            \textbf{Recurso} & \textbf{Horas utilizadas} & \textbf{Coste por hora (€)} & \textbf{Coste total (€)} \\ \hline
            CPU (c2-standard-8) & 600 & 0.43 & 258 \\
            GPU (NVIDIA K80 1 GPU) & 600 & 0.42 & 252 \\ \hline
            \textbf{Total} & \textbf{600} & \textbf{0.85} & \textbf{510} \\ \hline
        \end{tabular}
    \end{adjustbox}
    \caption{Coste estimado de los recursos computacionales.}
\end{table}


\subsection{Software y licencias}\label{subsec:software-y-licencias}
Para este proyecto, todas las herramientas de desarrollo utilizadas han sido de \textbf{código abierto}, por lo que no
se han generado costes asociados a licencias de software.

\subsection{Otros gastos}\label{subsec:otros-gastos}
Se han considerado gastos adicionales, como el uso de \textbf{conexión a internet} y el consumo de
\textbf{electricidad} durante el desarrollo y entrenamiento de los modelos.

\begin{table}[htp]\label{tab:otros-gastos}
    \centering
    \begin{tabular}{|l|c|}
        \hline
        \textbf{Concepto} & \textbf{Coste estimado (€)} \\ \hline
        Conexión a internet & 50 \\
        Electricidad & 40 \\ \hline
        \textbf{Total} & \textbf{90 €} \\ \hline
    \end{tabular}
    \caption{Otros gastos estimados.}
\end{table}


\subsection{Presupuesto total}\label{subsec:presupuesto-total}
Sumando los costes de mano de obra, los recursos computacionales y otros gastos, el presupuesto total estimado es el
siguiente:

\begin{table}[htp]\label{tab:presupuesto-total}
    \centering
    \begin{tabular}{|l|c|}
        \hline
        \textbf{Concepto} & \textbf{Coste (€)} \\ \hline
        Mano de obra & 8.000 \\
        Recursos computacionales & 510 \\
        Otros gastos & 90 \\ \hline
        \textbf{Total} & \textbf{8.600 €} \\ \hline
    \end{tabular}
    \caption{Presupuesto total estimado del proyecto.}
\end{table}
