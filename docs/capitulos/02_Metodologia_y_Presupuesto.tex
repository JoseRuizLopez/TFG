\chapter{Metodología y Presupuesto}\label{ch:metodologia-y-presupuesto}
\section{Metodología}\label{sec:metodologia}
La metodología adoptada para el desarrollo de este proyecto ha estado inspirada en el marco ágil
\textbf{Scrum}~\cite{noauthor_scrum_nodate}, una estrategia ampliamente utilizada en proyectos que requieren
flexibilidad, adaptabilidad y entregas incrementales.
Dado que un trabajo de investigación como este demanda constantes ajustes en función de los hallazgos y resultados
obtenidos, Scrum ha sido el marco adecuado para organizar el desarrollo de manera iterativa y colaborativa, facilitando
la mejora continua durante el proceso.

\subsection{Metodología Ágil Inspirada en Scrum}\label{subsec:metodologia-agil-inspirada-en-scrum}
El desarrollo del proyecto se organizó en \textbf{sprints}, que consistían en ciclos cortos de
trabajo (de dos semanas, generalmente), donde se definían objetivos concretos al inicio y se revisaban los avances al
final de cada sprint.
Esto permitió adaptar el trabajo de acuerdo con los descubrimientos y problemas emergentes a lo largo del proyecto.

\subsection{Gestión de Tareas}
\label{subsec:gestion-de-tareas}
Para mantener un control eficiente del progreso del proyecto, se utilizaron herramientas de gestión de tareas de forma
personal como \textbf{Notion}~\cite{noauthor_notion_nodate}, donde se organizaban las actividades de cada sprint.
Estas plataformas permitieron visualizar el estado de cada tarea y establecer plazos de entrega, promoviendo un flujo
de trabajo estructurado y ágil.\\[6pt]

Además, el uso de \textbf{Git}~\cite{chacon_pro_2014} como herramienta para el control de versiones fue clave en la
gestión de los avances de desarrollo.
Git permitió mantener un registro preciso de cada cambio en el código y los algoritmos, facilitando la colaboración con
el tutor y garantizando una trazabilidad eficiente de los ajustes realizados durante cada iteración.
Gracias a Git, se pudo revisar, deshacer y modificar el código de forma ágil, integrando nuevos enfoques o mejoras de
manera controlada tras cada sprint.\\[6pt]

Asimismo, se empleó \textbf{GitHub} como repositorio central del código, facilitando la colaboración y el control de
versiones.
Gracias a esta herramienta, fue posible llevar un historial detallado de los cambios, realizar revisiones de código y
gestionar ramas de desarrollo de manera efectiva.

\subsection{Ciclos de Trabajo}\label{subsec:ciclos-de-trabajo}
El desarrollo del proyecto se llevó a cabo mediante un enfoque iterativo basado en la metodología \textit{Scrum}, con
el propósito de garantizar una evolución progresiva del sistema y una adaptación eficiente a los desafíos encontrados
durante su implementación.
Para ello, el trabajo se organizó en \textbf{sprints}, cada uno con una duración aproximada de dos semanas.\\[6pt]

Cada sprint se estructuró en diferentes fases que permitieron planificar, ejecutar, evaluar y documentar los avances de
manera sistemática.
A continuación, se describen en detalle cada una de estas fases:

\subsubsection{Planificación del Sprint}
Al inicio de cada sprint, se realizaba una sesión de planificación en la que se definían los objetivos específicos para
el período.
En esta etapa, se revisaban los avances del sprint anterior, se priorizaban tareas pendientes y se establecían nuevos
hitos de acuerdo con las necesidades del proyecto.
Esta planificación se basaba en un análisis del \textit{backlog}, asegurando que las actividades seleccionadas fueran
alcanzables dentro del tiempo estipulado.

\subsubsection{Desarrollo e Implementación}
Durante esta fase, se llevaba a cabo la ejecución de las tareas planificadas.
Dependiendo de los objetivos del sprint, esto podía incluir la implementación de nuevos módulos, la optimización de
algoritmos, la integración de componentes o la corrección de errores identificados en iteraciones previas.\\[6pt]

La estrategia de trabajo se basó en la implementación incremental, donde cada nueva funcionalidad se agregaba de manera
progresiva, permitiendo pruebas y validaciones continuas.

\subsubsection{Pruebas y Ajustes}
Una vez desarrolladas las funcionalidades previstas en cada sprint, se procedía a la fase de pruebas.
En este punto, se aplicaban diferentes estrategias de evaluación, como pruebas unitarias, pruebas de integración y
validaciones experimentales, con el fin de verificar el correcto funcionamiento del sistema y detectar posibles fallos
o áreas de mejora.\\[6pt]

Durante esta etapa, también se realizaban ajustes en los parámetros de los algoritmos y en la configuración del sistema
para optimizar el rendimiento.
En algunos casos, los resultados de las pruebas conducían a la identificación de nuevas tareas o mejoras que se
incorporaban en el \textit{backlog} para ser abordadas en sprints posteriores.

\subsubsection{Revisión y Análisis de Resultados}
Al finalizar cada sprint, se realizaba una revisión en la que se analizaban los resultados obtenidos.
Se comparaban los avances con los objetivos planteados en la fase de planificación y se identificaban los logros
alcanzados, así como los aspectos que requerían mayor atención en futuras iteraciones.
Esta revisión permitía evaluar la efectividad de las estrategias implementadas y ajustar el enfoque del proyecto en
función de los aprendizajes obtenidos.\\[6pt]

Además, se documentaban los principales hallazgos y se generaban representaciones visuales, como gráficas y tablas,
para facilitar la comparación de los algoritmos, las pruebas y los resultados, proporcionando una visión clara de
la evolución del desarrollo.

\subsubsection{Documentación}
Un aspecto fundamental del proceso fue la documentación continua de cada sprint.
Esto incluyó la elaboración de registros detallados sobre las modificaciones en el código, los cambios en la
arquitectura del sistema, los resultados de las pruebas y los ajustes realizados en la configuración.\\[6pt]

La documentación también abarcó la justificación de decisiones técnicas, lo que permitió mantener una trazabilidad
clara del desarrollo.
Esto fue particularmente útil para garantizar la reproducibilidad de los experimentos y facilitar la generación de la
memoria del proyecto.


\newpage
\subsection{Visualización del Progreso}\label{subsec:visualizacion-del-progreso}


\begin{figure}[htp]
\definecolor{sprint_color1}{RGB}{51,102,254} % Azul oscuro para Sprint 1
\definecolor{sprint_color2}{RGB}{38, 173, 70} % Verde oscuro para Sprint 2
\definecolor{fase_color1}{RGB}{153,204,254} % Azul claro para Fase 1
\definecolor{fase_color2}{RGB}{138, 222, 158} % Verde claro más oscuro para Fase 2
\renewcommand\sfdefault{phv}
\renewcommand\mddefault{mc}
\renewcommand\bfdefault{bc}
\sffamily
\noindent
\resizebox{\textwidth}{!}{
    \begin{ganttchart}[
        group label node/.append style={anchor=east}, % Alinea grupo a la derecha
        bar label node/.append style={anchor=east}, % Alinea fases a la derecha
        group left shift=0,
        canvas/.append style={fill=none, draw=black!5, line width=.75pt},
        hgrid style/.style={draw=black!5, line width=.75pt},
        vgrid={*1{draw=black!5, line width=.75pt}},
        today label font=\small\bfseries,
        title/.style={draw=none, fill=none},
        title label font=\bfseries\footnotesize,
        title label node/.append style={align=center}, % Centrado horizontal
        include title in canvas=false,
        bar/.append style={draw=none}, % Sin borde en las barras
        bar incomplete/.append style={fill=barblue},
        group right shift=0,
        group height=.5,
        group peaks tip position=0,
        group progress label font=\bfseries\small,
        link/.style={-latex, line width=1.5pt, linkred},
        link label font=\scriptsize\bfseries,
        link label node/.append style={below left=-2pt and 0pt},
        x unit=0.4cm,
        y unit title=0.6cm, % Controla la altura de la fila del título
      ]{1}{27}

      % Omitimos el título DAYS en el mecanismo estándar y dejamos solo los números
      \gantttitle{1}{1} \gantttitle{2}{1} \gantttitle{3}{1} \gantttitle{4}{1} \gantttitle{5}{1}
      \gantttitle{6}{1} \gantttitle{7}{1} \gantttitle{8}{1} \gantttitle{9}{1} \gantttitle{10}{1}
      \gantttitle{11}{1} \gantttitle{12}{1} \gantttitle{13}{1} \gantttitle{14}{1} \gantttitle{15}{1}
      \gantttitle{16}{1} \gantttitle{17}{1} \gantttitle{18}{1} \gantttitle{19}{1} \gantttitle{20}{1}
      \gantttitle{21}{1} \gantttitle{22}{1} \gantttitle{23}{1} \gantttitle{24}{1} \gantttitle{25}{1}
      \gantttitle{26}{1} \gantttitle{27}{1} \gantttitle{28}{1}
      \\ % Salto de línea después de los títulos

      % Sprint 1 alineado a la derecha - con color personalizado
      \ganttgroup[name=sprint1, group/.append style={fill=sprint_color1}]{Sprint 1}{1}{14} \\
      \ganttbar[name=s1fase1, bar/.append style={fill=fase_color1}]
        {Planificación del sprint \textbf{| Fase 1}}{1}{1} \\
      \ganttbar[name=s1fase2, bar/.append style={fill=fase_color1}]
        {Desarrollo e implementación de algoritmos \textbf{| Fase 2}}{2}{4} \\
      \ganttbar[name=s1fase3, bar/.append style={fill=fase_color1}]
        {Pruebas y ajustes \textbf{| Fase 3}}{5}{10} \\
      \ganttbar[name=s1fase4, bar/.append style={fill=fase_color1}]
        {Revisión y análisis de resultados \textbf{| Fase 4}}{11}{12} \\
      \ganttbar[name=s1fase5, bar/.append style={fill=fase_color1}]
        {Documentación de los algoritmos y resultados \textbf{| Fase 5}}{13}{14} \\[grid]

      % Sprint 2 alineado a la derecha - con color personalizado
      \ganttgroup[name=sprint2, group/.append style={fill=sprint_color2}]{Sprint 2}{15}{28} \\
      \ganttbar[name=s2fase1, bar/.append style={fill=fase_color2}]
        {Planificación del sprint\textbf{| Fase 1}}{15}{15} \\
      \ganttbar[name=s2fase2, bar/.append style={fill=fase_color2}]
        {Desarrollo e implementación de algoritmos \textbf{| Fase 2}}{16}{18} \\
      \ganttbar[name=s2fase3, bar/.append style={fill=fase_color2}]
        {Pruebas y ajustes \textbf{| Fase 3}}{19}{24} \\
      \ganttbar[name=s2fase4, bar/.append style={fill=fase_color2}]
        {Revisión y análisis de resultados \textbf{| Fase 4}}{25}{26} \\
      \ganttbar[name=s2fase5, bar/.append style={fill=fase_color2}]
        {Documentación de los algoritmos y resultados \textbf{| Fase 5}}{27}{28}
    \end{ganttchart}
}
\caption{
Diagrama de Gantt que ilustra la planificación del proyecto. Cada barra representa un sprint, incluyendo sus fases
de planificación, desarrollo, pruebas y revisión. Las líneas verticales indican el día en relación con el
cronograma establecido.
}\label{fig:gantt-diagram}
\end{figure}

En la Figura~\ref{fig:gantt-diagram} se presenta el \textit{Diagrama de Gantt} del proyecto, donde se detallan dos
sprints junto con sus respectivas fases.
Si bien el proyecto ha consistido en múltiples sprints debido a su naturaleza iterativa, en este diagrama se han
seleccionado únicamente dos como ejemplo representativo del proceso de desarrollo.
Esta representación permite visualizar la distribución del tiempo y las interdependencias entre las distintas etapas,
proporcionando un seguimiento claro del progreso alcanzado.

\subsection{Adaptación del Tiempo de los Sprints}\label{subsec:adaptacion-del-tiempo-de-los-sprints}
Debido a ciertos retrasos en el desarrollo del TFG, fue necesario ajustar la duración de los sprints para asegurarse de
que todas las fases y pruebas del proyecto se completaran a tiempo.
La flexibilidad de Scrum y el uso de Git para coordinar versiones y tareas pendientes permitieron reorganizar las
prioridades y garantizar que los objetivos del proyecto fueran alcanzados sin comprometer la calidad de los resultados.

\section{Presupuesto}\label{sec:presupuesto}
El presupuesto estimado incluye tanto el tiempo dedicado al proyecto (mano de obra) como los recursos computacionales y
otros gastos relacionados.
Aunque muchos recursos utilizados son gratuitos o de bajo coste, se ha realizado una estimación considerando el valor
del tiempo invertido y el uso de recursos computacionales.

\subsection{Mano de obra}\label{subsec:mano-de-obra}
El trabajo total estimado es de \textbf{400 horas}, repartidas de manera flexible a lo largo de los ciclos.
El coste por hora se ha estimado en \textbf{20 euros}~\cite{noauthor_salario_nodate}, lo cual incluye las tareas de
investigación, desarrollo de algoritmos, análisis de resultados y documentación.

\begin{table}[htp]\label{tab:mano-de-obra}
    \centering
    \begin{adjustbox}{width=\linewidth}
        \begin{tabular}{|l|c|c|c|}
            \hline
            \textbf{Ciclo de trabajo} & \textbf{Horas dedicadas} & \textbf{Coste por hora (€)} &
            \textbf{Coste total (€)} \\ \hline
            Planificación del sprint & 30 & 20 & 600 \\
            Desarrollo e implementación de algoritmos & 80 & 20 & 1.600 \\
            Pruebas y ajustes & 180 & 20 & 3.600 \\
            Revisión y análisis de resultados & 55 & 20 & 1.100 \\
            Documentación de algoritmos y resultados & 55 & 20 & 1.100 \\ \hline
            \textbf{Total} & \textbf{400 horas} & & \textbf{8.000 €} \\ \hline
        \end{tabular}
    \end{adjustbox}
    \caption{Coste estimado de la mano de obra basado en los ciclos de trabajo.}
\end{table}


\subsection{Recursos computacionales}\label{subsec:recursos-computacionales}
El proyecto ha requerido el uso de recursos computacionales para entrenar los modelos de aprendizaje profundo,
especialmente para evaluar la eficacia de los algoritmos en la reducción de datos. \\[6pt]

El tutor \textbf{Daniel Molina Cabrera} me puso en disposición el acceso a un servidor de investigación con
disponibilidad de GPU, de manera que no ha supuesto ningún coste.
Por ello, para suponer un coste estimado, se ha supuesto lo que nos costaria un servidor de
\textbf{Google Cloud}~\cite{noauthor_overview_nodate} de \textbf{Compute Engine}~\cite{noauthor_what_nodate} en
Bélgica. \\[6pt]

Los componentes equivalentes del servidor en Compute Engine serían:
\begin{itemize}
    \item Un \textbf{Intel Xeon E-2226G} equivaldría a un \textbf{c2-standard-8}, cuyo coste es de 0.43 EUR/h.
    \item Un \textbf{NVIDIA TITAN Xp} equivaldría a una \textbf{NVIDIA K80 1 GPU}, GDDR5 de 12 GB cuyo coste es de 0.42
EUR/h.
\end{itemize}

De manerá que los gastos estimados serían:
\begin{table}[htp]\label{tab:recursos-computacionales}
    \centering
    \begin{adjustbox}{width=\linewidth}
        \begin{tabular}{|l|c|c|c|}
            \hline
            \textbf{Recurso} & \textbf{Horas utilizadas} & \textbf{Coste por hora (€)} &
            \textbf{Coste total (€)} \\ \hline
            CPU (c2-standard-8) & 600 & 0.43 & 258 \\
            GPU (NVIDIA K80 1 GPU) & 600 & 0.42 & 252 \\ \hline
            \textbf{Total} & \textbf{600} & \textbf{0.85} & \textbf{510} \\ \hline
        \end{tabular}
    \end{adjustbox}
    \caption{Coste estimado de los recursos computacionales.}
\end{table}


\subsection{Software y licencias}\label{subsec:software-y-licencias}
Para este proyecto, todas las herramientas de desarrollo utilizadas han sido de \textbf{código abierto}, por lo que no
se han generado costes asociados a licencias de software.

\subsection{Otros gastos}\label{subsec:otros-gastos}
Se han considerado gastos adicionales, como el uso de \textbf{conexión a internet} y el consumo de
\textbf{electricidad} durante el desarrollo y entrenamiento de los modelos.

\begin{table}[htp]\label{tab:otros-gastos}
    \centering
    \begin{tabular}{|l|c|}
        \hline
        \textbf{Concepto} & \textbf{Coste estimado (€)} \\ \hline
        Conexión a internet & 50 \\
        Electricidad & 40 \\ \hline
        \textbf{Total} & \textbf{90 €} \\ \hline
    \end{tabular}
    \caption{Otros gastos estimados.}
\end{table}


\subsection{Presupuesto total}\label{subsec:presupuesto-total}
Sumando los costes de mano de obra, los recursos computacionales y otros gastos, el presupuesto total estimado es el
siguiente:

\begin{table}[htp]\label{tab:presupuesto-total}
    \centering
    \begin{tabular}{|l|c|}
        \hline
        \textbf{Concepto} & \textbf{Coste (€)} \\ \hline
        Mano de obra & 8.000 \\
        Recursos computacionales & 510 \\
        Otros gastos & 90 \\ \hline
        \textbf{Total} & \textbf{8.600 €} \\ \hline
    \end{tabular}
    \caption{Presupuesto total estimado del proyecto.}
\end{table}
