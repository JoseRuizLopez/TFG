% !TeX root = ../proyecto.tex

\chapter{Metodología y Presupuesto}\label{ch:metodologia-y-presupuesto}
\section{Metodología}\label{sec:metodologia}
Para organizar el desarrollo del proyecto se optó por una metodología inspirada en el marco ágil
\textbf{Scrum}~\cite{ScrumGuide}, ampliamente utilizado en entornos donde se requiere adaptación constante y
entregas progresivas.
Dado que el desarrollo del trabajo se ajustaba progresivamente según los resultados obtenidos en cada fase,
este enfoque demostró ser adecuado para mantener una estructura flexible y favorecer un avance iterativo

\subsection{Enfoque Ágil Basado en Scrum}\label{subsec:enfoque-agil-basado-en-scrum}
El proyecto se dividió en ciclos de trabajo breves (\textbf{sprints}), generalmente de dos semanas.
Cada sprint comenzaba con una planificación en la que se establecían objetivos concretos y finalizaba con una revisión
para evaluar el progreso y hacer ajustes si era necesario.
Esta dinámica permitió mantener un ritmo constante, al tiempo que se dejaba margen para adaptarse a posibles
imprevistos o nuevas ideas surgidas durante el desarrollo.

\subsection{Organización de Tareas}\label{subsec:organizacion-de-tareas}
Para la organización y seguimiento de tareas se empleó una herramienta digital de apoyo (Notion~\cite{NotionGestionTareas}),
que facilitó la planificación personal y visualización de actividades pendientes.


Este tipo de plataformas contribuyen a mejorar la organización y la colaboración en proyectos complejos,
tal como se ha demostrado en entornos profesionales mediante el uso de tecnologías de la información orientadas a la gestión colaborativa~\cite{EnhancingCollaborationProjectbased}.


Esta plataforma facilitó la visualización de las actividades pendientes y cumplidas, lo que ayudó a no perder de vista
los plazos y prioridades de cada fase.


En paralelo, se empleó \textbf{Git}~\cite{chaconProGit2014} para el control de versiones del código, lo que resultó
fundamental para mantener un seguimiento detallado de los cambios realizados en cada etapa del proyecto.
También se utilizó \textbf{GitHub} como repositorio central, facilitando así la colaboración con el tutor y permitiendo
una trazabilidad clara de todas las modificaciones.


\subsection{Ciclos de Trabajo}\label{subsec:ciclos-de-trabajo}
Los sprints se estructuraron en varias fases repetidas en cada iteración:

\begin{itemize}
    \item \textbf{Planificación}: En esta etapa se definían los objetivos del sprint, basándose en lo ya realizado y en
          las tareas pendientes más prioritarias.
          El análisis del \textbf{backlog} permitía seleccionar actividades realistas que se pudieran completar en el plazo
          previsto.

    \item \textbf{Desarrollo e implementación}: Se ejecutaban las tareas previstas, como la programación de nuevos
          módulos, mejoras en algoritmos o la integración de componentes.
          El enfoque fue incremental, es decir, se añadían funcionalidades poco a poco, asegurando que cada avance se pudiera
          probar por separado.

    \item \textbf{Pruebas y ajustes}: Una vez desarrolladas las funcionalidades, se realizaban pruebas (unitarias, de
          integración o empíricas) para validar el funcionamiento del sistema y ajustar los parámetros si fuese necesario.
          A veces, los resultados de esta fase daban lugar a nuevas tareas que se añadían al backlog.

    \item \textbf{Revisión y análisis de resultados}: Al finalizar el sprint, se revisaban los objetivos cumplidos, se
          analizaban los resultados obtenidos y se valoraba la necesidad de replantear enfoques.
          Esta revisión también ayudaba a identificar puntos de mejora y reforzar los que ya funcionaban bien.

    \item \textbf{Documentación}: Durante todo el proceso se fue registrando la evolución del proyecto, desde los
          cambios en el código hasta los resultados de las pruebas.
          Este esfuerzo permitió tener una memoria bien estructurada, facilitar la reproducción de experimentos y justificar cada
          decisión tomada.
\end{itemize}


\newpage
\subsection{Seguimiento del Progreso}\label{subsec:seguimiento-del-progreso}

\begin{figure}[htp]
    \definecolor{sprint_color1}{RGB}{51,102,254} % Azul oscuro para Sprint 1
    \definecolor{sprint_color2}{RGB}{38, 173, 70} % Verde oscuro para Sprint 2
    \definecolor{fase_color1}{RGB}{153,204,254} % Azul claro para Fase 1
    \definecolor{fase_color2}{RGB}{138, 222, 158} % Verde claro más oscuro para Fase 2
    \renewcommand\sfdefault{phv}
    \renewcommand\mddefault{mc}
    \renewcommand\bfdefault{bc}
    \sffamily
    \noindent
    \resizebox{\textwidth}{!}{
        \begin{ganttchart}[
                group label node/.append style={anchor=east}, % Alinea grupo a la derecha
                bar label node/.append style={anchor=east}, % Alinea fases a la derecha
                group left shift=0,
                canvas/.append style={fill=none, draw=black!5, line width=.75pt},
                hgrid style/.style={draw=black!5, line width=.75pt},
                vgrid={*1{draw=black!5, line width=.75pt}},
                today label font=\small\bfseries,
                title/.style={draw=none, fill=none},
                title label font=\bfseries\footnotesize,
                title label node/.append style={align=center}, % Centrado horizontal
                include title in canvas=false,
                bar/.append style={draw=none}, % Sin borde en las barras
                bar incomplete/.append style={fill=barblue},
                group right shift=0,
                group height=.5,
                group peaks tip position=0,
                group progress label font=\bfseries\small,
                link/.style={-latex, line width=1.5pt, linkred},
                link label font=\scriptsize\bfseries,
                link label node/.append style={below left=-2pt and 0pt},
                x unit=0.4cm,
                y unit title=0.6cm, % Controla la altura de la fila del título
            ]{1}{27}

            % Omitimos el título DAYS en el mecanismo estándar y dejamos solo los números
            \gantttitle{1}{1} \gantttitle{2}{1} \gantttitle{3}{1} \gantttitle{4}{1} \gantttitle{5}{1}
            \gantttitle{6}{1} \gantttitle{7}{1} \gantttitle{8}{1} \gantttitle{9}{1} \gantttitle{10}{1}
            \gantttitle{11}{1} \gantttitle{12}{1} \gantttitle{13}{1} \gantttitle{14}{1} \gantttitle{15}{1}
            \gantttitle{16}{1} \gantttitle{17}{1} \gantttitle{18}{1} \gantttitle{19}{1} \gantttitle{20}{1}
            \gantttitle{21}{1} \gantttitle{22}{1} \gantttitle{23}{1} \gantttitle{24}{1} \gantttitle{25}{1}
            \gantttitle{26}{1} \gantttitle{27}{1} \gantttitle{28}{1}
            \\ % Salto de línea después de los títulos

            % Sprint 1 alineado a la derecha - con color personalizado
            \ganttgroup[name=sprint1, group/.append style={fill=sprint_color1}]{Sprint 1}{1}{14} \\
            \ganttbar[name=s1fase1, bar/.append style={fill=fase_color1}]
            {Planificación del sprint \textbf{| Fase 1}}{1}{1} \\
            \ganttbar[name=s1fase2, bar/.append style={fill=fase_color1}]
            {Desarrollo e implementación de algoritmos \textbf{| Fase 2}}{2}{4} \\
            \ganttbar[name=s1fase3, bar/.append style={fill=fase_color1}]
            {Pruebas y ajustes \textbf{| Fase 3}}{5}{10} \\
            \ganttbar[name=s1fase4, bar/.append style={fill=fase_color1}]
            {Revisión y análisis de resultados \textbf{| Fase 4}}{11}{12} \\
            \ganttbar[name=s1fase5, bar/.append style={fill=fase_color1}]
            {Documentación de los algoritmos y resultados \textbf{| Fase 5}}{13}{14} \\[grid]

            % Sprint 2 alineado a la derecha - con color personalizado
            \ganttgroup[name=sprint2, group/.append style={fill=sprint_color2}]{Sprint 2}{15}{28} \\
            \ganttbar[name=s2fase1, bar/.append style={fill=fase_color2}]
            {Planificación del sprint\textbf{| Fase 1}}{15}{15} \\
            \ganttbar[name=s2fase2, bar/.append style={fill=fase_color2}]
            {Desarrollo e implementación de algoritmos \textbf{| Fase 2}}{16}{18} \\
            \ganttbar[name=s2fase3, bar/.append style={fill=fase_color2}]
            {Pruebas y ajustes \textbf{| Fase 3}}{19}{24} \\
            \ganttbar[name=s2fase4, bar/.append style={fill=fase_color2}]
            {Revisión y análisis de resultados \textbf{| Fase 4}}{25}{26} \\
            \ganttbar[name=s2fase5, bar/.append style={fill=fase_color2}]
            {Documentación de los algoritmos y resultados \textbf{| Fase 5}}{27}{28}
        \end{ganttchart}
    }
    \caption{
        Diagrama de Gantt que muestra la planificación de dos sprints del proyecto, con sus respectivas fases: planificación,
        desarrollo, pruebas, revisión y documentación. Este esquema representa la estructura iterativa adoptada a lo largo del desarrollo
    }\label{fig:gantt-diagram}
\end{figure}

Para tener una visión clara del avance del proyecto, se elaboró un \textbf{diagrama de Gantt}~[\ref{fig:gantt-diagram}]
que recogía dos sprints como ejemplo del proceso general.
En él se reflejan las distintas fases mencionadas, así como la duración aproximada de cada una.
Aunque se realizaron más sprints a lo largo del trabajo, este diagrama sirve como muestra del esquema de planificación
utilizado.

\subsection{Ajuste de los Tiempos}\label{subsec:ajuste-de-los-tiempos}
Durante el desarrollo surgieron algunos retrasos que obligaron a reajustar los tiempos previstos para los sprints.
Gracias a la flexibilidad que permite \textbf{Scrum} y al uso de herramientas como \textbf{Git},
permitió reorganizar prioridades y redistribuir tareas sin comprometer los objetivos del proyecto sin comprometer la calidad del trabajo.

\section{Presupuesto}\label{sec:presupuesto}
El presupuesto estimado incluye tanto el tiempo dedicado al proyecto (mano de obra) como los recursos computacionales y
otros gastos relacionados.
Aunque muchos recursos utilizados son gratuitos o de bajo coste, se ha realizado una estimación considerando el valor
del tiempo invertido y el uso de recursos computacionales.

\subsection{Mano de obra}\label{subsec:mano-de-obra}
El trabajo total estimado es de \textbf{400 horas}, repartidas de manera flexible a lo largo de los ciclos.
El coste por hora se ha estimado en \textbf{20 euros}~\cite{SalarioParaData}, lo cual incluye las tareas de
investigación, desarrollo de algoritmos, análisis de resultados y documentación.

\begin{table}[htp]\label{tab:mano-de-obra}
    \centering
    \begin{adjustbox}{width=\linewidth}
        \begin{tabular}{|l|c|c|c|}
            \hline
            \textbf{Ciclo de trabajo}                 & \textbf{Horas dedicadas} & \textbf{Coste por hora (€)} &
            \textbf{Coste total (€)}                                                                                              \\ \hline
            Planificación del sprint                  & 30                       & 20                          & 600              \\
            Desarrollo e implementación de algoritmos & 80                       & 20                          & 1.600            \\
            Pruebas y ajustes                         & 180                      & 20                          & 3.600            \\
            Revisión y análisis de resultados         & 55                       & 20                          & 1.100            \\
            Documentación de algoritmos y resultados  & 55                       & 20                          & 1.100            \\ \hline
            \textbf{Total}                            & \textbf{400 horas}       &                             & \textbf{8.000 €} \\ \hline
        \end{tabular}
    \end{adjustbox}
    \caption{Estimación del coste de la mano de obra distribuido por fases del desarrollo del proyecto.}
\end{table}


\subsection{Recursos computacionales}\label{subsec:recursos-computacionales}
El proyecto ha requerido el uso de recursos computacionales para entrenar los modelos de aprendizaje profundo,
especialmente para evaluar la eficacia de los algoritmos en la reducción de datos.


El tutor \textbf{Daniel Molina Cabrera} me puso en disposición el acceso a un servidor de investigación con
disponibilidad de GPU, de manera que no ha supuesto ningún coste.


Aunque el acceso al servidor de investigación con GPU no implicó coste real,
se ha realizado una estimación hipotética utilizando precios de \textbf{Google Cloud}~\cite{OverviewGoogleCloud}
usando un \textbf{Compute Engine}~\cite{WhatCloudRun} en Bélgica para valorar el uso de recursos.


Los componentes equivalentes del servidor en Compute Engine serían:
\begin{itemize}
    \item Un \textbf{Intel Xeon E-2226G} equivaldría a un \textbf{c2-standard-8}, cuyo coste es de 0.43 EUR/h.
    \item Un \textbf{NVIDIA TITAN Xp} equivaldría a una \textbf{NVIDIA K80 1 GPU}, GDDR5 de 12 GB cuyo coste es de 0.42
          EUR/h.
\end{itemize}

De modo que los gastos estimados serían:
\begin{table}[htp]\label{tab:recursos-computacionales}
    \centering
    \begin{adjustbox}{width=\linewidth}
        \begin{tabular}{|l|c|c|c|}
            \hline
            \textbf{Recurso}       & \textbf{Horas utilizadas} & \textbf{Coste por hora (€)} &
            \textbf{Coste total (€)}                                                                        \\ \hline
            CPU (c2-standard-8)    & 600                       & 0.43                        & 258          \\
            GPU (NVIDIA K80 1 GPU) & 600                       & 0.42                        & 252          \\ \hline
            \textbf{Total}         & \textbf{600}              & \textbf{0.85}               & \textbf{510} \\ \hline
        \end{tabular}
    \end{adjustbox}
    \caption{Simulación del coste de recursos computacionales equivalentes en Google Cloud.}
\end{table}


\subsection{Software y licencias}\label{subsec:software-y-licencias}
Para este proyecto, todas las herramientas de desarrollo utilizadas han sido de \textbf{código abierto},
por lo que no se han incurrido en costes de licencias.

\subsection{Gastos indirectos}\label{subsec:gastos-indirectos}
Se han considerado gastos adicionales, como el uso de \textbf{conexión a internet} y el consumo de
\textbf{electricidad} durante el desarrollo y entrenamiento de los modelos.

\begin{table}[htp]\label{tab:gastos-indirectos}
    \centering
    \begin{tabular}{|l|c|}
        \hline
        \textbf{Concepto}   & \textbf{Coste estimado (€)} \\ \hline
        Conexión a internet & 50                          \\
        Electricidad        & 40                          \\ \hline
        \textbf{Total}      & \textbf{90 €}               \\ \hline
    \end{tabular}
    \caption{Estimación de gastos indirectos asociados al desarrollo del proyecto.}
\end{table}


\subsection{Presupuesto total}\label{subsec:presupuesto-total}
Sumando los costes de mano de obra, los recursos computacionales y otros gastos, el presupuesto total estimado es el
siguiente:

\begin{table}[htp]\label{tab:presupuesto-total}
    \centering
    \begin{tabular}{|l|c|}
        \hline
        \textbf{Concepto}        & \textbf{Coste (€)} \\ \hline
        Mano de obra             & 8.000              \\
        Recursos computacionales & 510                \\
        Otros gastos             & 90                 \\ \hline
        \textbf{Total}           & \textbf{8.600 €}   \\ \hline
    \end{tabular}
    \caption{Resumen del presupuesto total estimado, incluyendo mano de obra, recursos y gastos adicionales.}
\end{table}

\newpage
En conjunto, la planificación metodológica y la estimación de recursos permitieron garantizar el avance
ordenado del proyecto y anticipar su viabilidad técnica y económica.