\chapter{Metodología y Presupuesto}\label{ch:metodologia-y-presupuesto}
En este capítulo se describe la metodología utilizada para gestionar el desarrollo del proyecto, así como el
presupuesto estimado.
Se ha optado por una \textbf{metodología ágil}, adaptada a la naturaleza iterativa y flexible del desarrollo de un
proyecto de investigación como este.
Además, se presenta el presupuesto que refleja los recursos empleados, tanto en tiempo como en infraestructura.

\section{Metodología}\label{sec:metodologia}
El enfoque elegido para la gestión del proyecto ha sido la \textbf{metodología ágil}, que permite gestionar proyectos
de forma incremental e iterativa.
Dado que el desarrollo de un trabajo de investigación implica constantes ajustes y mejoras en cada etapa, la
flexibilidad de la metodología ágil ha sido clave para adaptarse a los nuevos descubrimientos y desafíos a lo largo del
proyecto. \\[6pt]

Para gestionar el proyecto de manera efectiva, se utilizó \textbf{Git}~\cite{} para el control de versiones del código
y el transpaso fácil de datos entre el servidor y el alumno.

\subsection{Metodología Ágil}\label{subsec:metodologia-agil}
La metodología ágil se caracteriza por su capacidad para organizar el trabajo en ciclos cortos, cada uno con
objetivos específicos y entregas incrementales.
Para este proyecto, la \textbf{planificación}, el \textbf{desarrollo}, la \textbf{investigación} y el
\textbf{análisis de resultados} no se realizaron como fases completamente separadas, sino que se intercalaron durante
todo el ciclo del proyecto.

\subsection{Fases iterativas}\label{subsec:fases-iterativas}
A diferencia de una planificación tradicional secuencial, en este proyecto se ha trabajado en varias fases de forma
simultánea y en ciclos iterativos, con los objetivos y prioridades ajustándose a medida que avanzaba el trabajo.

\begin{enumerate}
    \item \textbf{Planificación e investigación continua}:
    \begin{itemize}
        \item En lugar de una planificación inicial fija, se realizó una planificación continua.
Los objetivos de cada iteración se establecían al inicio de cada ciclo de trabajo, según las necesidades y avances
previos.
        \item La investigación sobre los algoritmos y técnicas de reducción de datos fue un proceso continuo,
intercalado con el desarrollo y análisis de los resultados obtenidos en las distintas iteraciones.
    \end{itemize}

    \item \textbf{Desarrollo iterativo de los algoritmos}:
    \begin{itemize}
        \item La implementación de los diferentes algoritmos (aleatorio, búsqueda local y genéticos) no fue una única
fase, sino que se intercaló con la investigación y análisis.
Los algoritmos se desarrollaron en incrementos, permitiendo una mejora continua basada en la retroalimentación y
pruebas constantes.
        \item Los algoritmos se evaluaron y ajustaron constantemente, lo que permitió hacer mejoras tempranas y
corregir posibles fallos durante el ciclo de desarrollo.
    \end{itemize}

    \item \textbf{Análisis incremental de resultados}:
    \begin{itemize}
        \item El análisis de los resultados se realizó tras cada ciclo de pruebas, lo que permitió evaluar el impacto
de los cambios implementados en los algoritmos y ajustar la planificación para las siguientes iteraciones.
        \item Este enfoque incremental permitió identificar patrones y problemas a medida que avanzaba el proyecto, lo
que facilitó la optimización de los algoritmos, planificación de nuevas pruebas y la reducción de datos de manera más
eficiente.
    \end{itemize}
\end{enumerate}

\subsection{Ciclos de trabajo}\label{subsec:ciclos-de-trabajo}
El trabajo se organizó en ciclos de aproximadamente 2 semanas cada uno, donde se definían metas concretas,
como la implementación de un algoritmo o la evaluación de un conjunto de resultados.
Al final de cada ciclo, se realizaba una revisión de los avances y se ajustaban los objetivos del siguiente ciclo,
además de realizar una reunión con el tutor para orientar las nuevas pruebas y los siguientes avances. \\[6pt]

Cada ciclo estaba formado por:
\begin{enumerate}
    \item \textbf{Investigación} inicial y \textbf{diseño} preliminar del algoritmo.
    \item \textbf{Implementación} básica del algoritmo y \textbf{evaluación} inicial.
    \item \textbf{Ajustes} en los parámetros del algoritmo y \textbf{pruebas} comparativas.
    \item \textbf{Optimización} y \textbf{análisis} final de resultados.
    \item \textbf{Documentación} del algoritmo y de los resultados obtenidos.
\end{enumerate}

Cada ciclo permitiría la entrega de un incremento funcional del proyecto, con un enfoque en mantener la flexibilidad
para adaptarse a cualquier descubrimiento nuevo.

Pero tras atrasarse el desarrollo del TFG, se tuvo que adaptar el tiempo de cada ciclo para que diese tiempo a
realizar todas las pruebas necesarias.

\section{Presupuesto}\label{sec:presupuesto}
El presupuesto estimado incluye tanto el tiempo dedicado al proyecto (mano de obra) como los recursos computacionales y
otros gastos relacionados.
Aunque muchos recursos utilizados son gratuitos o de bajo coste, se ha realizado una estimación considerando el valor
del tiempo invertido y el uso de recursos computacionales.

\subsection{Mano de obra}\label{subsec:mano-de-obra}
El trabajo total estimado es de \textbf{400 horas}, repartidas de manera flexible a lo largo de los ciclos.
El coste por hora se ha estimado en \textbf{20 euros}, lo cual incluye las tareas de investigación, desarrollo de
algoritmos, análisis de resultados y documentación.

\begin{table}[htp]\label{tab:mano-de-obra}
    \centering
    \begin{adjustbox}{width=\linewidth}
        \begin{tabular}{|l|c|c|c|}
            \hline
            \textbf{Fase} & \textbf{Horas dedicadas} & \textbf{Coste por hora (€)} & \textbf{Coste total (€)} \\ \hline
            Planificación e investigación & 60 & 20 & 1.200 \\
            Desarrollo de algoritmos & 180 & 20 & 3.600 \\
            Análisis de resultados & 80 & 20 & 1.600 \\
            Redacción y documentación & 80 & 20 & 1.600 \\ \hline
            \textbf{Total} & \textbf{400 horas} & & \textbf{8.000 €} \\ \hline
        \end{tabular}
    \end{adjustbox}
    \caption{Coste estimado de la mano de obra.}
\end{table}


\subsection{Recursos computacionales}\label{subsec:recursos-computacionales}
El proyecto ha requerido el uso de recursos computacionales para entrenar los modelos de aprendizaje profundo,
especialmente para evaluar la eficacia de los algoritmos en la reducción de datos. \\[6pt]

El tutor \textbf{Daniel Molina Cabrera} me puso en disposición el acceso a un servidor de investigación con
disponibilidad de GPU, de manera que no ha supuesto ningún coste.
Por ello, para suponer un coste estimado, se ha supuesto lo que nos costaria un servidor de
\textbf{Google Cloud}~\cite{} de \textbf{Compute Engine}~\cite{} en Bélgica. \\[6pt]

Los componentes equivalentes del servidor en Compute Engine serían:
\begin{itemize}
    \item Un \textbf{Intel Xeon E-2226G} equivaldría a un \textbf{c2-standard-8}, cuyo coste es de 0.43 EUR/h.
    \item Un \textbf{NVIDIA TITAN Xp} equivaldría a una \textbf{NVIDIA K80 1 GPU}, GDDR5 de 12 GB cuyo coste es de 0.42
EUR/h.
\end{itemize}

De manerá que los gastos estimados serían:
\begin{table}[htp]\label{tab:recursos-computacionales}
    \centering
    \begin{adjustbox}{width=\linewidth}
        \begin{tabular}{|l|c|c|c|}
            \hline
            \textbf{Recurso} & \textbf{Horas utilizadas} & \textbf{Coste por hora (€)} & \textbf{Coste total (€)} \\ \hline
            CPU (c2-standard-8) & 210 & 0.43 & 90.3 \\
            GPU (NVIDIA K80 1 GPU) & 210 & 0.42 & 88.2 \\ \hline
            \textbf{Total} & \textbf{210} & \textbf{0.85} & \textbf{178.5} \\ \hline
        \end{tabular}
    \end{adjustbox}
    \caption{Coste estimado de los recursos computacionales.}
\end{table}


\subsection{Software y licencias}\label{subsec:software-y-licencias}
Para este proyecto, todas las herramientas de desarrollo utilizadas han sido de \textbf{código abierto}, por lo que no
se han generado costes asociados a licencias de software.

\subsection{Otros gastos}\label{subsec:otros-gastos}
Se han considerado gastos adicionales, como el uso de \textbf{conexión a internet} y el consumo de
\textbf{electricidad} durante el desarrollo y entrenamiento de los modelos.

\begin{table}[htp]\label{tab:otros-gastos}
    \centering
    \begin{tabular}{|l|c|}
        \hline
        \textbf{Concepto} & \textbf{Coste estimado (€)} \\ \hline
        Conexión a internet & 50 \\
        Electricidad & 40 \\ \hline
        \textbf{Total} & \textbf{90 €} \\ \hline
    \end{tabular}
    \caption{Otros gastos estimados.}
\end{table}


\subsection{Presupuesto total}\label{subsec:presupuesto-total}
Sumando los costes de mano de obra, los recursos computacionales y otros gastos, el presupuesto total estimado es el
siguiente:

\begin{table}[htp]\label{tab:presupuesto-total}
    \centering
    \begin{tabular}{|l|c|}
        \hline
        \textbf{Concepto} & \textbf{Coste (€)} \\ \hline
        Mano de obra & 8.000 \\
        Recursos computacionales & 178.5 \\
        Otros gastos & 90 \\ \hline
        \textbf{Total} & \textbf{8.268,5 €} \\ \hline
    \end{tabular}
    \caption{Presupuesto total estimado del proyecto.}
\end{table}


\section{Conclusión}\label{sec:conclusion}
La utilización de una \textbf{metodología ágil} ha permitido gestionar este proyecto de manera flexible y eficiente,
favoreciendo la adaptación continua a medida que surgían nuevos retos y descubrimientos.
La intercalación de las fases de planificación, desarrollo, investigación y análisis ha garantizado una entrega
incremental y un ajuste continuo del trabajo, asegurando la calidad y la optimización de los resultados. \\[6pt]

El presupuesto total estimado, de \textbf{8.268,5 euros}, refleja principalmente los costes asociados a la mano de obra y
los recursos computacionales necesarios para el desarrollo de los modelos y la ejecución de los experimentos.
