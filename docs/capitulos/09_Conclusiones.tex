% !TeX root = ../proyecto.tex

\chapter{Conclusiones}\label{ch:conclusiones}
\section{Resumen general del trabajo}\label{sec:resumen-general}

Este Trabajo de Fin de Grado aborda el problema de la reducción de conjuntos de datos en el contexto del aprendizaje profundo,
proponiendo el uso de algoritmos meméticos como técnica de selección de instancias.
A lo largo del desarrollo se diseñan, implementan y evalúan diferentes enfoques evolutivos (genéticos y meméticos),
con el objetivo de identificar subconjuntos representativos de imágenes que permitan entrenar modelos convolucionales de forma eficiente y sin pérdida significativa de rendimiento.

El estudio se centra en evaluar cómo diferentes variantes algorítmicas impactan sobre la precisión, la estabilidad y el tamaño final del conjunto de entrenamiento.
Para ello, se utilizan dos modelos de referencia ampliamente extendidos (ResNet50 y MobileNetV2) y dos conjuntos de datos con distintos niveles de complejidad: Rock, Paper, Scissors y PAINTING.

\section{Principales contribuciones}\label{sec:contribuciones}

Los resultados experimentales permiten extraer varias conclusiones clave:

\begin{itemize}
    \item El uso de algoritmos evolutivos mejora notablemente la calidad de los subconjuntos generados en comparación con estrategias aleatorias o deterministas.
          Incluso las versiones básicas, como el algoritmo genético clásico, superan con claridad al enfoque aleatorio.

    \item Las mejoras introducidas en los operadores de cruce y mutación (ponderación basada en fitness y mutación adaptativa)
          incrementan la estabilidad y precisión de los resultados, especialmente en escenarios con porcentajes bajos de datos iniciales.

    \item La inclusión del reinicio poblacional ofrece cierto beneficio en términos de diversidad,
          aunque no aporta mejoras significativas respecto a las versiones mejoradas del genético en cuanto a precisión media.

    \item Las versiones \textit{libres} de los algoritmos, donde el tamaño del subconjunto se ajusta dinámicamente,
          permiten alcanzar un equilibrio más eficaz entre precisión y tamaño final.
          Este enfoque demuestra ser especialmente útil cuando no se conoce a priori la fracción óptima de datos para entrenar.

    \item El algoritmo memético, y en particular su versión libre (MEM-F), alcanza los mejores resultados globales.
          Este enfoque logra una elevada precisión, una notable reducción del conjunto de datos y una gran estabilidad entre ejecuciones,
          consolidándose como la solución más eficaz del estudio.
\end{itemize}


\section{Validación y generalización}\label{sec:validacion-generalizacion}
La validación con un segundo conjunto de datos (PAINTING), más complejo y con mayor número de clases, permite comprobar la capacidad de generalización de los enfoques propuestos.
El algoritmo memético libre no solo mantiene su rendimiento, sino que en algunos casos supera a la precisión obtenida con el 100\,\% del conjunto completo.
Además, se confirma que las soluciones generadas preservan una distribución equilibrada de clases y ajustan el tamaño del subconjunto según la dificultad del problema.


\section{Implicaciones y aplicaciones futuras}\label{sec:implicaciones-aplicaciones}
Los resultados obtenidos demuestran que es posible reducir de forma significativa los conjuntos de datos para entrenar modelos convolucionales sin comprometer el rendimiento,
iempre que se utilicen estrategias de selección guiadas y adaptativas.
Esto tiene implicaciones relevantes en escenarios donde los recursos computacionales son limitados o donde la auditoría y trazabilidad de los datos es una exigencia normativa.

Como líneas futuras, se plantea:
\begin{itemize}
    \item Extender el estudio a otros modelos de visión por computador más recientes y especializados.
    \item Aplicar las técnicas propuestas en tareas más complejas, como segmentación semántica o clasificación multietiqueta.
    \item Investigar la incorporación de criterios adicionales en la función de fitness, como el tiempo de entrenamiento o el consumo energético.
    \item Explorar la integración de los algoritmos meméticos con otras técnicas de reducción de datos como distilación de conocimiento o poda de redes.
\end{itemize}


\section{Reflexión final}\label{sec:reflexion-final}
Este trabajo demuestra que la combinación de técnicas evolutivas y búsqueda local, aplicada de forma flexible y adaptativa,
constituye una herramienta poderosa para optimizar el entrenamiento de modelos de aprendizaje profundo.
Más allá de la mejora en precisión o eficiencia, esta línea de investigación apunta hacia un aprendizaje más sostenible, trazable y accesible,
en línea con las necesidades actuales de la inteligencia artificial moderna.

Desde un punto de vista personal, este proyecto ha representado un desafío enriquecedor tanto a nivel técnico como organizativo.
La implementación completa de los algoritmos, la gestión eficiente del entorno experimental y
la interpretación rigurosa de los resultados me han permitido consolidar conocimientos clave en aprendizaje automático y optimización.
Asimismo, la dificultad de compaginar este trabajo con responsabilidades laborales ha reforzado mi compromiso con el rigor y la constancia,
valores que considero fundamentales en mi futuro profesional.