\chapter{Desarrollo Experimental}\label{ch:desarrollo-experimental}

Tras la explicación del contenido necesario para el entendendimiento y desarrollo del TFG en los anteriores capítulos,
en este capítulo se prcedera a explicar las pruebas y posteriores mejoras que se han realizado y los posteriores
resultados obtenidos. \\[6pt]

Las pruebas iniciales que se plantearon fueron tomar un dataset simple para realizar las primeras pruedas, y ya cuando
funcionase correctamente, probar con otro dataset más complejo o realista.
Para ello se decidió usar el dataset de \textbf{RPS}. \\[6pt]

Para obtener unos primeros resultados con este dataset, se planteó usar el modelo de \textbf{Resnet50}.
Se hicieron unas pruebas con distintos porcentajes para ver distintos resultados posibles.

\begin{table}[htp]
    \centering
    \resizebox{\textwidth}{!}{
        \begin{tabular}{lp{2cm}lp{2cm}p{2cm}p{2cm}p{2cm}p{2.2cm}}
            \toprule
            \textbf{Algoritmo} & \textbf{Porcentaje Inicial} & \textbf{Duracion} & \textbf{Accuracy (Avg)} &
            \textbf{Precision (Avg)} & \textbf{Recall (Avg)} & \textbf{F1-score (Avg)} &
            \textbf{Evaluaciones Realizadas} \\
            \midrule
            aleatorio & 10 & 00:45:08 & 76,55\% & 81,80\% & 76,55\% & 76,25\% & 100 \\
            aleatorio & 20 & 01:10:27 & 81,77\% & 84,70\% & 81,77\% & 81,59\% & 100 \\
            aleatorio & 50 & 02:24:49 & 87,14\% & 88,09\% & 87,14\% & 86,97\% & 100 \\
            aleatorio & 100 & 00:02:42 & 87,90\% & 88,96\% & 87,90\% & 87,81\% & 1 \\
            \bottomrule
        \end{tabular}
    }
    \caption{Resultados de la generación inicial con \textbf{Resnet50}}
    \label{tab:initial-generation-resnet}
\end{table}

Fijandonos en la tabla~\ref{tab:initial-generation-resnet}, las duraciones que se indican son lo que tarda en
ejecutarse todas las evaluaciones que se indican.
Vemos que son tiempos muy grandes, por ello se decidió probar el algoritmo \textbf{MobileNet} para agilizar el proceso.

\begin{table}[htp]
    \centering
    \resizebox{\textwidth}{!}{
        \begin{tabular}{lp{2cm}lp{2cm}p{2cm}p{2cm}p{2cm}p{2.2cm}}
            \toprule
        \textbf{Algoritmo} & \textbf{Porcentaje Inicial} & \textbf{Duracion} & \textbf{Accuracy (Avg)} &
        \textbf{Precision (Avg)} & \textbf{Recall (Avg)} & \textbf{F1-score (Avg)} & \textbf{Evaluaciones Realizadas} \\
        \midrule
        aleatorio & 10 & 00:29:29 & 72,31\% & 76,40\% & 72,31\% & 69,62\% & 100 \\
        aleatorio & 20 & 00:50:36 & 76,48\% & 78,82\% & 76,48\% & 75,58\% & 100 \\
        aleatorio & 50 & 01:54:09 & 75,56\% & 79,72\% & 75,56\% & 74,67\% & 100 \\
        aleatorio & 100 & 00:02:12 & 76,08\% & 79,97\% & 76,08\% & 75,61\% & 1 \\
        \bottomrule
        \end{tabular}
    }
    \caption{Resultados de la generación inicial con \textbf{MobileNet}}
    \label{tab:initial-generation-mobilenet}
\end{table}



\section{Datasets}\label{sec:datasets}
En el aprendizaje profundo, los datasets son colecciones de datos etiquetados o no etiquetados que se utilizan para
entrenar modelos.
Estos conjuntos de datos suelen contener ejemplos organizados que representan la entrada para el modelo, y en muchos
casos, también las etiquetas correspondientes que indican la salida deseada.
Los datasets varían en tamaño, calidad y tipo, dependiendo de la tarea a resolver, como la clasificación de imágenes,
el reconocimiento de patrones o la predicción de series temporales.

\subsection{Rock, Paper, Scissors (Piedra, Papel, Tijera)}\label{subsec:rock-paper-scissors}
\textbf{Rock, Paper, Scissors}~\cite{Rock Paper Scissors Dataset} fue creado originalmente por Laurence Moroney y se
utiliza para clasificar imágenes de las manos representando los gestos de `piedra', `papel' y `tijeras'.
El conjunto de datos contiene alrededor de 2,500 imágenes distribuidas en tres categorías: piedra, papel y tijeras.
Las imágenes están en color y tienen un tamaño de 300x300 píxeles. \\[6pt]

En este trabajo, se ha utilizado el dataset de \textbf{Rock, Paper, Scissors} para evaluar el rendimiento del modelo en
un problema de clasificación de imágenes más variado y natural, que involucra múltiples clases.
Además, permite explorar la eficacia de los algoritmos meméticos en un entorno más cercano al reconocimiento de
objetos, lo que añade mayor complejidad al problema.

\subsection{MNIST (Modified National Institute of Standards and Technology)}\label{subsec:mnist}
\textbf{MNIST}~\cite{MNIST Dataset} es uno de los datasets más utilizados en el campo del aprendizaje automático y el
aprendizaje profundo.
Contiene 70,000 imágenes de dígitos escritos a mano (60,000 para el conjunto de entrenamiento y 10,000 para el de
prueba).
Las imágenes son en escala de grises y tienen un tamaño de 28x28 píxeles, con cada píxel representando una intensidad
de color entre 0 (negro) y 255 (blanco).

Este dataset se utiliza comúnmente como \textbf{benchmark} para evaluar modelos de clasificación de imágenes,
particularmente en arquitecturas convolucionales.

La simplicidad de \textbf{MNIST} lo hace ideal para probar modelos de redes neuronales convolucionales, ya que ofrece un
equilibrio entre un problema fácil de entender, pero con suficiente complejidad para que los modelos más avanzados
demuestren mejoras.

\subsection{Comparación con otros datasets}\label{subsec:comparacion-con-otros-datasets}
La selección de estos dos datasets responde a la necesidad de evaluar los algoritmos meméticos en distintos niveles de
complejidad.
\textbf{MNIST}, con imágenes en escala de grises de bajo nivel de complejidad, proporciona una referencia clara y
estandarizada para comparar el rendimiento y la reducción de datos.
Por otro lado, el dataset de \textbf{Rock, Paper, Scissors} introduce más desafíos visuales y complejidades,
permitiendo analizar cómo los algoritmos meméticos se comportan en escenarios más complejos que podrían ser
representativos de aplicaciones más reales en visión por computadora.
