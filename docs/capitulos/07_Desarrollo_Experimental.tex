\chapter{Desarrollo Experimental}\label{ch:desarrollo_experimental}

En este capítulo, tras haber explicado todo el conocimiento necesario para desarrollar el TFG, se explicarán todas las
pruebas y mejores que se han realizado, junto a los resultados obtenidos. \\[6pt]


Las pruebas iniciales que se plantearon fueron tomar un dataset simple para realizar las primeras pruedas, y ya cuando
funcionase correctamente, probar con otro dataset más complejo o realista.
Para ello se decidió usar el dataset de \textbf{RPS}. \\[6pt]

Para obtener unos primeros resultados con este dataset, se planteó usar el modelo de \textbf{Resnet50}.
Se hicieron unas pruebas con distintos porcentajes para ver cuál era el porcentaje que más merecia la pena usar.

\begin{table}[htp]
    \centering
    \resizebox{\textwidth}{!}{
        \begin{tabular}{lp{2cm}lp{2cm}p{2cm}p{2cm}p{2cm}p{2.2cm}}
            \toprule
            \textbf{Algoritmo} & \textbf{Porcentaje Inicial} & \textbf{Duracion} & \textbf{Accuracy (Avg)} &
            \textbf{Precision (Avg)} & \textbf{Recall (Avg)} & \textbf{F1-score (Avg)} &
            \textbf{Evaluaciones Realizadas} \\
            \midrule
            aleatorio & 10 & 00:45:08 & 76,55 & 81,80 & 76,55 & 76,25 & 100 \\
            aleatorio & 20 & 01:10:27 & 81,77 & 84,70 & 81,77 & 81,59 & 100 \\
            aleatorio & 50 & 02:24:49 & 87,14 & 88,09 & 87,14 & 86,97 & 100 \\
            aleatorio & 100 & 00:02:42 & 87,90 & 88,96 & 87,90 & 87,81 & 1 \\
            \bottomrule
        \end{tabular}
    }
    \caption{}
    \label{tab:initial_generation_resnet}
\end{table}


Tras realizar las pruebas iniciales, se pensó en probar con otro modelo, uno más veloz para que tardase menos tiempo en
realizar pruebas y poder realizar más pruebas en menos tiempo.
Para ello se decido usar \textbf{Mobilenet}, debido a que es más rápido y suele tener resultados parecidos si
los Dataset no son muy grandes.
