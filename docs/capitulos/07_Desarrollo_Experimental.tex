\chapter{Desarrollo Experimental}\label{ch:desarrollo_experimental}

Las pruebas iniciales que se plantearon fueron tomar un dataset simple para realizar las primeras pruedas, y ya cuando
funcionase correctamente, probar con otro dataset más complejo o realista.
Para ello se decidió usar el dataset de \textbf{RPS}~\cite{}, ya que tiene un número pequeño de imagenes, 2520 en el
\textbf{training set} y 3 clases (840 imágenes por clase), y 372 imagenes para el \textbf{test set} (124 por clase).
\\[6pt]

Para obtener unos primeros resultados con este dataset, se planteó usar el modelo de \textbf{Resnet50}~\cite{}.
Se hicieron unas pruebas con distintos porcentajes para ver cuál era el porcentaje que más merecia la pena usar.
\\[6pt]

Tras realizar las pruebas iniciales, se pensó en probar con otro modelo, uno más veloz para que tardase menos tiempo en
realizar pruebas y poder realizar más pruebas en menos tiempo.
Para ello se decido usar \textbf{Mobilenet}~\cite{}, debido a que es más rápido y suele tener resultados parecidos si
los Dataset no son muy grandes ~\cite{ResnetVsMobilenet}.
