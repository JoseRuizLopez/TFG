% !TeX root = ../proyecto.tex

% Archivo: config.tex
% Configuraciones generales, paquetes y definiciones

\documentclass[a4paper,11pt]{book}

\makeatletter
\def\input@path{{./config/}{./capitulos/}{./portada/}{./prefacios/}}
\makeatother

\usepackage{listings}
\usepackage[utf8]{inputenc}
\usepackage[spanish]{babel}
\decimalpoint
\usepackage{dcolumn}
\usepackage{fancyhdr}
\setlength{\headheight}{25.3pt}
\addtolength{\topmargin}{-7.3pt}
\usepackage{graphicx}
\usepackage{afterpage}
\usepackage[colorlinks=false, pdfborder={0 0 1}, pdfborderstyle={/S/U/W 1}]{hyperref}
\usepackage{colortbl,longtable}
\usepackage[stable]{footmisc}
\usepackage{verbatim}
\usepackage{datetime}
\usepackage{booktabs}
\usepackage{adjustbox}
\usepackage{breakurl}
\usepackage{pgfgantt} % Paquete para diagramas de Gantt
\usepackage[T1]{fontenc}
\usepackage[utf8]{inputenc}
\usepackage{lmodern}

\PassOptionsToPackage{hyphens}{url}

\usepackage[backend=biber, defernumbers=true, citestyle=numeric-comp, bibstyle=ieee, sorting=none]{biblatex}

\DeclareBibliographyCategory{cited}
\AtEveryCitekey{\addtocategory{cited}{\thefield{entrykey}}}
% Configurando BibLaTeX
\DefineBibliographyStrings{spanish}{
  url = {URL},
  andothers={et ~al\adddot}
}
\addbibresource{bibliografia/bibliografia.bib}
% Incluyo todas las referencias en la bibliografía para ver las que no se han citado
\nocite{*}

% Columnas con punto decimal español
\newcolumntype{.}{D{.}{\esperiod}{-1}}
\makeatletter
\addto\shorthandsspanish{\let\esperiod\es@period@code}
\makeatother

\addto\captionsspanish{\renewcommand{\tablename}{Tabla}}


% Información reutilizable
\newcommand{\myTitle}{Algoritmos memeticos para reducir datos de entrenamiento en modelos de aprendizaje profundo convolucionales\xspace}
\newcommand{\myDegree}{Grado en Ingeniería Informática\xspace}
\newcommand{\myName}{JOSE RUIZ LOPEZ (alumno)\xspace}
\newcommand{\myProf}{DANIEL MOLINA CABRERA (tutor)\xspace}
\newcommand{\myFaculty}{Escuela Técnica Superior de Ingenierías Informática y de Telecomunicación\xspace}
\newcommand{\myDepartment}{Departamento de Ciencias de la Computación e Inteligencia Artificial\xspace}
\newcommand{\myUni}{\protect{Universidad de Granada}\xspace}
\newcommand{\myLocation}{Granada\xspace}
\newcommand{\myTime}{\today\xspace}
\newcommand{\myVersion}{Version 0.1\xspace}


% Configuración de hyperlinks
\hypersetup{
  pdfauthor = {\myName —(ruizlopezjose@correo.ugr.es)},
  pdftitle = {\myTitle},
  pdfsubject = {Trabajo Fin de Grado},
  pdfkeywords = {Algoritmos memeticos, Imagenes, Modelos de Aprendizaje profundo convolucionales},
  pdfcreator = {LaTeX con el paquete PDFLatex y Biber},
  pdfproducer = {PDFLatex},
  breaklinks=true
}

% Definición de teoremas y otros entornos
\newtheorem{teorema}{Teorema}[chapter]
\newtheorem{ejemplo}{Ejemplo}[chapter]
\newtheorem{definicion}{Definición}[chapter]

% Configuración de listings para código
\lstset{
  frame=Ltb, framerule=0.5pt, aboveskip=0.5cm, framextopmargin=3pt,
  framexbottommargin=3pt, framexleftmargin=0.1cm, framesep=0pt, rulesep=.4pt,
  backgroundcolor=\color{gray97}, rulesepcolor=\color{black},
  stringstyle=\ttfamily, showstringspaces=false, basicstyle=\scriptsize\ttfamily,
  commentstyle=\color{gray45}, keywordstyle=\bfseries, numbers=left,
  numbersep=6pt, numberstyle=\tiny, breaklines=true
}


% Definiciones adicionales
\newcommand{\HRule}{\rule{\linewidth}{0.5mm}}

\pagestyle{fancy}
\fancyhf{}
\fancyhead[LO]{\leftmark}
\fancyhead[RE]{\rightmark}
\fancyhead[RO,LE]{\textbf{\thepage}}
\renewcommand{\chaptermark}[1]{\markboth{\textbf{#1}}{}}
\renewcommand{\sectionmark}[1]{\markright{\textbf{\thesection. #1}}}

\setlength{\headheight}{1.5\headheight}
\setlength{\parskip}{6pt}       % espacio vertical del salto de linea
\setlength{\parindent}{15pt}    % espacio de la sangría

\newcommand{\monthnamecaps}[1]{%
  \ifcase#1
  \or Enero%
  \or Febrero%
  \or Marzo%
  \or Abril%
  \or Mayo%
  \or Junio%
  \or Julio%
  \or Agosto%
  \or Septiembre%
  \or Octubre%
  \or Noviembre%
  \or Diciembre%
  \fi}

\newdateformat{mesanyo}{\monthnamecaps{\THEMONTH} de \THEYEAR}