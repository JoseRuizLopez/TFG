% Archivo: config.tex
% Configuraciones generales, paquetes y definiciones

\documentclass[a4paper,11pt]{book}
\usepackage{listings}
\usepackage[utf8]{inputenc}
\usepackage[spanish]{babel}
\decimalpoint
\usepackage{dcolumn}
\usepackage{fancyhdr}
\usepackage{graphicx}
\usepackage{afterpage}
\usepackage[pdfborder={000}]{hyperref}
\usepackage{url}
\usepackage{colortbl,longtable}
\usepackage[stable]{footmisc}
\usepackage{verbatim}
\usepackage{booktabs}
\usepackage{adjustbox}

% Columnas con punto decimal español
\newcolumntype{.}{D{.}{\esperiod}{-1}}
\makeatletter
\addto\shorthandsspanish{\let\esperiod\es@period@code}
\makeatother

\addto\captionsspanish{\renewcommand{\tablename}{Tabla}}


% Información reutilizable
\newcommand{\myTitle}{Algoritmos meméticos para reducir datos de entrenamiento en modelos de aprendizaje profundo convolucionales\xspace}
\newcommand{\myDegree}{Grado en Ingeniería Informática\xspace}
\newcommand{\myName}{José Ruiz López (alumno)\xspace}
\newcommand{\myProf}{Daniel Molina Cabrera (tutor)\xspace}
\newcommand{\myFaculty}{Escuela Técnica Superior de Ingenierías Informática y de Telecomunicación\xspace}
\newcommand{\myDepartment}{Departamento de Ciencias de la Computación e Inteligencia Artificial\xspace}
\newcommand{\myUni}{\protect{Universidad de Granada}\xspace}
\newcommand{\myLocation}{Granada\xspace}
\newcommand{\myTime}{\today\xspace}
\newcommand{\myVersion}{Version 0.1\xspace}


% Configuración de hyperlinks
\hypersetup{
    pdfauthor = {\myName (email (en) ugr (punto) es)},
    pdftitle = {\myTitle},
    pdfsubject = {},
    pdfkeywords = {palabra_clave1, palabra_clave2, palabra_clave3, ...},
    pdfcreator = {LaTeX con el paquete ....},
    pdfproducer = {pdflatex}
}

% Definición de teoremas y otros entornos
\newtheorem{teorema}{Teorema}[chapter]
\newtheorem{ejemplo}{Ejemplo}[chapter]
\newtheorem{definicion}{Definición}[chapter]

% Configuración de listings para código
\lstset{
    frame=Ltb, framerule=0.5pt, aboveskip=0.5cm, framextopmargin=3pt,
    framexbottommargin=3pt, framexleftmargin=0.1cm, framesep=0pt, rulesep=.4pt,
    backgroundcolor=\color{gray97}, rulesepcolor=\color{black},
    stringstyle=\ttfamily, showstringspaces=false, basicstyle=\scriptsize\ttfamily,
    commentstyle=\color{gray45}, keywordstyle=\bfseries, numbers=left,
    numbersep=6pt, numberstyle=\tiny, breaklines=true
}

% Definiciones adicionales
\newcommand{\HRule}{\rule{\linewidth}{0.5mm}}

\pagestyle{fancy}
\fancyhf{}
\fancyhead[LO]{\leftmark}
\fancyhead[RE]{\rightmark}
\fancyhead[RO,LE]{\textbf{\thepage}}
\renewcommand{\chaptermark}[1]{\markboth{\textbf{#1}}{}}
\renewcommand{\sectionmark}[1]{\markright{\textbf{\thesection. #1}}}

\setlength{\headheight}{1.5\headheight}
